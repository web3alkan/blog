\documentclass[a4paper,12pt]{article}
\usepackage[T1]{fontenc}
\usepackage[turkish]{babel}
\usepackage[utf8]{inputenc}
\usepackage{csquotes}
\usepackage{hyperref}
\usepackage{graphicx}
\usepackage{amssymb,amsmath}
\usepackage{pdflscape}
\usepackage[style=chicago-notes,backend=biber]{biblatex}
\addbibresource{kaynakca.bib}

\begin{titlepage}
\begin{center}

\begin{center}
\textbf{\Large Merkez Bankası Dijital Paralarının
Programlanabilirliği:}\\[0.4em]
\textbf{\Large TCMB Dijital Türk Lirası Üzerindeki Akıllı
Sözleşmelerin Hukuki Niteliği}
\end{center}


\Large Bilgi ve İletişim Teknolojileri Dersi\\[2cm]
\large ASBÜ Bilişim ve Teknoloji Hukuku Tezli Yüksek Lisans Programı\\[3cm]

\textbf{\large Hazırlayan:}\\
\textbf{\large Tarık İ. ALKAN}\\[0.5cm]

\textbf{\large Dersin hocası:}\\
\textbf{\large Dr. Ömer Fatih SAYAN}\\[3cm]

\includegraphics[width=3.5cm]{logo.png} \\[0.5cm]  % Üniversite logosu eklenecekse

Ankara Sosyal Bilimler Üniversitesi \\
2025

\end{center}
\end{titlepage}



\begin{document}



\begin{abstract}
Bu çalışma, merkez bankası dijital paralarının (CBDC) teknik, hukuki ve anayasal yönlerini analiz etmekte ve Türkiye için önerilen dijital Türk lirası altyapısına akıllı sözleşmelerin entegrasyonunu değerlendirmektedir. Blokzincir teknolojisi, dağıtık defter sistemleri ve programlanabilir para kavramları incelenmiş; Avrupa Birliği, Çin, ABD, Hindistan ve Singapur örnekleri karşılaştırmalı olarak ele alınmıştır.

Senaryo analizi üzerinden kamu ihalelerinde Dijital TL ile yapılan bir ödemenin yanlış işleyişi sonucu doğan uyuşmazlık analiz edilerek, algoritmik sistemlerin klasik hukuk ilkeleriyle nasıl çatışabileceği gösterilmiştir. Son bölümde, algoritmik adalet, kodun hukuki niteliği ve hukuk devleti ilkesi bağlamında CBDC altyapılarında olması gereken denge tartışılmış, Türkiye için somut politika önerileri sunulmuştur.

\vspace{1em}

\textit{Anahtar Kelimeler: CBDC, Akıllı Sözleşmeler, Dijital TL, Hukuk Devleti, Algoritmik Adalet, Kodun Normatifliği}
\end{abstract}

\renewcommand{\abstractname}{Abstract}
\begin{abstract}
This thesis analyzes the legal and constitutional implications of central bank digital currencies (CBDCs), with a specific focus on integrating smart contracts into the Digital Turkish Lira infrastructure. The study explores blockchain systems, distributed ledger technologies, and the concept of programmable money. Case studies from the European Union, China, the United States, India, and Singapore are comparatively examined.

Using a scenario-based analysis, the paper demonstrates how automatic payment execution via smart contracts in public procurement may lead to legal disputes. The discussion expands to include algorithmic justice, the normative power of code, and the rule of law, offering policy recommendations tailored for the Turkish legal and administrative context.

\vspace{1em}

\textit{Keywords: CBDC, Smart Contracts, Digital TL, Rule of Law, Algorithmic Justice, Code as Law}
\end{abstract}

\newpage


\section{Giriş}

Finansal sistemlerin dijitalleşmesi, yalnızca ekonomik araçları değil, aynı zamanda bu araçların işleyişini düzenleyen hukuki yapıları da dönüştürmektedir. Bu dönüşümün en çarpıcı örneklerinden biri, merkez bankalarının giderek daha fazla ilgi göstermeye başladığı dijital para birimleridir. \textit{Merkez Bankası Dijital Paraları} (\textit{Central Bank Digital Currencies – CBDCs}), yalnızca yeni bir ödeme aracı değil, aynı zamanda para politikasının uygulanma biçimini, finansal aracılık sistemini ve bireylerin ekonomik özerkliklerini etkileyebilecek potansiyele sahiptir. CBDC’lerin, özellikle \textit{blokzincir} (blockchain) tabanlı \textit{akıllı sözleşmeler} (smart contracts) ile entegre edilerek “programlanabilir para” (\textit{programmable money}) niteliği kazanması, bu dönüşümün yalnızca teknik bir evrim değil, aynı zamanda normatif bir kırılma yarattığını göstermektedir. Bu yeni bağlamda hukuk, sadece düzenleyici değil, aynı zamanda bu teknolojilerin varlık koşullarını da şekillendiren bir alan haline gelmiştir.

Akıllı sözleşmelerin CBDC’lerle entegrasyonu, ödeme işlemlerinin önceden belirlenen koşullara bağlanmasını ve belirli olaylara otomatik olarak tepki veren kod parçacıkları aracılığıyla finansal akışın yönetilmesini mümkün kılmaktadır. Örneğin, sosyal yardım ödemelerinin yalnızca belirli ürün veya hizmetler için harcanabilmesi ya da bir kira sözleşmesinin ödeme yapılmadığında otomatik olarak feshedilmesi gibi senaryolar, artık teknik olarak uygulanabilir durumdadır. Bu durum, hukukun temel ilkeleri açısından ciddi sorular doğurmaktadır. İşlemlerin tamamen otomatikleşmesi, işlemlere insan müdahalesi olmaksızın sonuç doğurma kabiliyeti kazanması, kişilerin yargı yoluna erişim hakkını nasıl etkileyecektir? Geri döndürülemez işlem mantığı, hata, hile veya zorlayıcı nedenler gibi klasik hukuki müdahale gerekçelerini geçersiz mi kılacaktır?  

Bu sorular, özellikle Lawrence Lessig’in ileri sürdüğü “\textit{code is law}” yaklaşımı çerçevesinde daha da kritik bir hale gelmektedir. Lessig’e göre dijital ortamlarda bireylerin davranışlarını sınırlayan temel faktör yalnızca hukuki normlar değil, yazılımın kendisidir; yani “kod”, modern toplumda normatif bir otorite işlevi üstlenmektedir.\footcite{lessig1999code} Bu görüş, programlanabilir paralarla birlikte daha önce hiç olmadığı kadar somut bir anlam kazanmıştır. Paranın işleyişi kod ile belirlendiğinde, hukukun müdahale alanı daralmakta; egemenlik kavramı yazılım geliştiricilerinin iradesine teslim edilebilmektedir. Amerika Birleşik Devletleri Temsilciler Meclisi’nde 2023 yılında sunulan “CBDC Anti-Surveillance State Act” tasarısı, tam da bu türden yeni nesil kod temelli otoritelerin demokratik süreçleri kötü etkileyebileceği endişesiyle gündeme gelmiştir.\footcite{emmer2023cbdc}  

Bu çalışmanın yanıt bulmaya çalıştığı en büyük soru, mevcut hukuk sistemlerinin ve düzenleyici yapıların, dağıtık yapıya sahip, otomatik karar alma mekanizmalarını içeren bir CBDC sistemine ne ölçüde uyum sağlayabileceğidir. Çalışma, hem Türkiye özelinde hem de Avrupa Birliği, Çin, ABD, Singapur ve Hindistan gibi ülkelerin deneyimlerinden hareketle karşılaştırmalı bir çerçeve sunmayı amaçlamaktadır. Bu ülkelerde 2022 sonrası dönemde hızlanan CBDC girişimleri ve dijital para hukuku tartışmaları, akıllı sözleşmelerin teknik imkânları ile hukukun sınırları arasında giderek belirginleşen gerilimi gözler önüne sermektedir.\footcite{orchid2023project}

Makalede kullanılan yöntem, bilişim hukuku, sermaye piyasası hukuku ve bilgisayar bilimleri kesişiminde konumlanan disiplinler arası bir yaklaşımı benimsemektedir. Literatür taraması ile birlikte kavramsal senaryo analizi ve karşılaştırmalı hukuk tekniği bir arada kullanılmıştır. Böylelikle yalnızca normatif düzenin mevcut hali değil, teknolojik imkânların sunduğu potansiyel ve bunun hukukla olan ilişkisi detaylı biçimde irdelenmektedir. “Code is law” ilkesi, otomatik ifa sistemleri, geri döndürülemez işlemler, algoritmik özerklik, kamu gücünün yazılım vasıtasıyla icrası gibi konular merkeze alınarak, CBDC'nin hukuki sınırları eleştirel bir perspektiften değerlendirilecektir.


\section{Blokzincir ve Akıllı Sözleşmelerin Temel Kavramları}

Merkez bankası dijital paralarının hukukî çerçevesini analiz edebilmek için öncelikle bu sistemlerin altyapısını oluşturan teknolojilerin temel özelliklerinin kavranması gerekmektedir. Özellikle \textit{blokzincir} (blockchain) ve \textit{akıllı sözleşmeler} (smart contracts), dijital paraların programlanabilirliği açısından belirleyici unsurlar haline gelmiştir. Bu bölümde, söz konusu teknolojilerin yapısal özellikleri ve hukuki analiz bakımından doğurdukları sorular ele alınacaktır.

Blokzincir, genel olarak bir \textit{dağıtık defter teknolojisi} (Distributed Ledger Technology – DLT) biçiminde tanımlanır. İlk kez 2008 yılında Satoshi Nakamoto adlı/kod adlı kişi ya da kişiler tarafından Bitcoin teknik dokümanında önerilen bu yapı, merkezi olmayan bir veri kaydının tüm ağ katılımcıları (\textit{node}) tarafından tutulmasını esas alır.\footcite{nakamoto2008bitcoin} Her bir işlem bloğu, kendinden önce gelen bloğun kriptografik özetini (\textit{hash}) içerdiğinden, zincirin herhangi bir halkasına müdahale edilmesi tüm sistemi etkileyecek şekilde kurgulanmıştır. Bu özellik, verinin değiştirilemezliği (\textit{immutability}) ve şeffaflığı gibi prensipler doğrultusunda işlem güvenliğini temin eder. Blokzincirin en kritik özelliği, sistem içindeki güven ilişkisinin yazılımla, yani kodla sağlanmasıdır. Bu durum, noterlik sistemi örneğinde olduğu gibi klasik hukuk düzenlerinde "güven" kavramının yerine "algoritmik denetimi" ikame eder. Hukuki anlamda sorulması gereken ise şudur: Kod tarafından sağlanan bu güven, normatif teminatların yerini almalı mı ve alabilir mi?

Blokzincir üzerine inşa edilen sistemlerin en önemli unsurlarından biri de akıllı sözleşmelerdir. İlk kez 1994 yılında Nick Szabo tarafından teorik olarak formüle edilen bu kavram, \textit{koşullu otomasyon yeteneği taşıyan yazılım protokolleri} şeklinde tanımlanabilir.\footcite{szabo1997formalizing} Teknik olarak bir akıllı sözleşme, belirli koşullar gerçekleştiğinde otomatik olarak çalışan ve genellikle Ethereum gibi \textit{Turing-complete} blokzincir platformlarında çalışan bir koddur (üzerinde çalışan akıllı sözleşmelerin herhangi bir hesaplanabilir işlemi gerçekleştirebilecek kadar güçlü). Kod blokzincire yüklendikten sonra değiştirilemez hale gelir ve şeffaf biçimde tüm kullanıcılar tarafından incelenebilir. Akıllı sözleşmelerin bu özerk yapısı, onları yalnızca teknik araçlar olmaktan çıkarıp hukukî sorumluluk tartışmalarının da merkezine yerleştirmiştir. Örneğin, sigorta sektöründe uygulanan senaryolarda uçuş gecikmeleri gibi olaylar harici veri sağlayıcılar (\textit{oracles}) vasıtasıyla tespit edilmekte ve bu veriye göre ödeme otomatik biçimde gerçekleşmektedir. Bu durumda insan müdahalesi olmadan işlem tesis edilmekte, klasik hukukta öngörülen irade beyanı ve değerlendirme süreçleri tamamen ortadan kalkmaktadır.

Ancak akıllı sözleşmelerin her zaman hukukî anlamda geçerli bir sözleşme olduğu söylenemez. Zira Borçlar Hukuku açısından bir sözleşmenin oluşması için tarafların karşılıklı ve birbirine uygun irade beyanında bulunması, ehliyet sahibi olması ve konu ile şekil bakımından geçerlilik şartlarını yerine getirmesi gerekir. Akıllı sözleşmelerin birçoğu bu şartları taşımaz; hatta çoğu durumda tarafların gerçek kimlikleri dahi (belirlenebilir oluşu bir yana) belirli değildir. 

Akıllı sözleşmelerin değiştirilemez ve aracısız niteliği, sistemsel sorumluluk açısından da yeni tartışmalar yaratmaktadır. Özellikle 2016 yılında Ethereum platformunda meydana gelen ve “The DAO” olarak bilinen olayda, kod hatası nedeniyle milyonlarca dolar değerinde kripto varlık farklı bir hesaba aktarıldı. Bu olayın ardından Ethereum ağı “hard fork” adı verilen bir protokolle ikiye ayrılarak, kodun geriye dönük biçimde değiştirilmesi sağlandı.\footcite{mehar2019dao} Bu gelişme, “code is law” söyleminin mutlak bir ilke olarak kabul edilemeyeceğini, topluluk iradesinin de teknolojiye yön verebileceğini göstermiştir.

Dünya genelinde merkez bankalarının dijital para projelerinde akıllı sözleşmeleri nasıl ele aldığına bakıldığında, teknik esnekliğin hukukî düzenlemelerle uyumlu hale getirilmesinin temel bir ihtiyaç olduğu görülmektedir. Avrupa Merkez Bankası’nın 2023 tarihli Dijital Euro Paketi’nde bu teknolojiye yer verilmekle birlikte, programlanabilirliğin kamusal fayda ile dengelenmesi gerektiği vurgulanmıştır.\footcite{ecb2023digital} Benzer şekilde Singapur Para Otoritesi’nin Project Orchid kapsamında yürüttüğü çalışmalarda da, CBDC uygulamalarında akıllı sözleşme güvenliği, dış veri sağlayıcıların doğruluğu ve işlemlerin denetlenebilirliği gibi kriterler ön plana çıkarılmıştır.\footcite{mas2023orchid} Bu bölümde ele alınan kavramlarla, ilerleyen kısımlarda Dijital Türk Lirası özelinde yapılacak normatif analizlerin temeli atılmış oldu.


\section{Merkez Bankası Dijital Paraları (CBDC): Tanım, Amaç ve Teknik Mimariler}

Merkez bankası dijital paraları, bir ülkenin merkez bankası tarafından ihraç edilen ve yasal para biriminin kriptografik dijital biçimini temsil eden değer birimleridir. Temel farkları, doğrudan kamu otoritesinin teminatı altında bulunmaları ve merkez bankasının bilançosunda yer alan bir varlık olarak işlem görmeleridir. Bu yönleriyle, ticari bankalar ya da özel ödeme hizmeti sağlayıcıları tarafından sunulan elektronik para çözümlerinden açık biçimde ayrılırlar. Zira CBDC’ler, özel sektörün sunduğu e-cüzdanlar, ön ödemeli kartlar veya mobil ödeme platformlarının aksine, \textbf{nihai ödeme aracı} olarak kabul edilir ve sistemde yer alan diğer tüm dijital araçların likiditeye dönüşümünde güvenli dayanak noktası oluştururlar.\footcite{bis2023cbdc} Kripto paralarla benzer bir dijital altyapı üzerinde işlem görseler de, aralarındaki fark gerek ihraç otoritesinin niteliği (yani tamamiyle merkezi olması), gerekse değer istikrarı bakımından belirgin (stabil) olmasıdır. Bitcoin gibi kripto varlıklar merkezi olmayan sistemlerde, algoritmik ya da arz sınırlı yapılarla çalışırken; CBDC’ler doğrudan devlet güvencesine sahiptir ve ulusal para biriminin nominal değeri ile bire bir eşitlik taşır.

CBDC’lerin geliştirilme gerekçeleri elbette teknolojik yeniliklere tepki verme olarak basite indirgenemez. Birçok merkez bankası, bu sistemleri finansal altyapının yeniden düzenlenmesine yönelik stratejik bir araç olarak görmektedir. Özellikle finansal kapsayıcılığın artırılması, bu alandaki temel motivasyonlardan biridir. Gelişmekte olan ülkelerde ciddi bir nüfusun banka hesabına erişimi bulunmamakta ve bu durum, dijital ödeme sistemlerinden dışlanma riskini doğurmaktadır. CBDC'ler sayesinde bireylerin doğrudan merkez bankası garantili bir sistem aracılığıyla ödeme yapabilmeleri ve para saklayabilmeleri mümkün hale gelmektedir. Ayrıca, küresel ölçekte gözlenen nakit kullanımındaki azalma, merkez bankası parasının kamusal yaşamda görünürlüğünü zayıflatmakta; bu eğilim karşısında dijital düzlemde yeni bir merkez bankası parası inşa edilmesi kaçınılmaz hale gelmektedir.\footcite{ecb2023digital} Bunun yanı sıra özel sektör tarafından sunulan dijital ödeme çözümlerinin zamanla tekelleşmesi, merkez bankalarının parasal egemenlik açısından müdahale ihtiyacını doğurmuştur. CBDC’ler, bu bağlamda rekabeti teşvik eden, sistemsel dayanıklılığı artıran ve kamu çıkarını önceleyen bir alternatif sunmaktadır. Para politikasının aktarım mekanizmalarının etkinliği de CBDC tartışmalarında öne çıkan bir başka husustur. Dijital para yoluyla merkez bankalarının hanehalkına veya işletmelere doğrudan müdahalesi, özellikle kriz dönemlerinde hızlı ve hedefli parasal genişleme politikalarının uygulanmasına olanak tanıyabilir.\footcite{mas2023orchid}

CBDC’lerin teknik mimarisi ise erişim tipi, değer temsili ve dağıtım modeli gibi temel boyutlar üzerinden şekillenmektedir. Erişim açısından CBDC’ler genel olarak iki türde tasarlanır: toptan (wholesale) ve perakende (retail) modeller. Toptan model yalnızca lisanslı finansal kurumların erişimine açıktır ve büyük hacimli ödemelerin mutabakatı için kullanılır. Buna karşılık perakende model, doğrudan bireylerin ve işletmelerin erişimine açılmıştır ve adeta dijital bir “nakit para” gibi işlev görür. Çin’in e-CNY uygulaması, Hindistan’ın Dijital Rupi sistemi ve Jamaika’nın Jam-Dex projesi bu kapsamda örnek teşkil etmektedir.\footcite{imf2023retail} Teknik olarak bu dijital paralar, değer temsili bakımından hesap tabanlı ya da token tabanlı olarak kurgulanabilir. Hesap tabanlı sistemlerde kullanıcılar merkez bankasında veya yetkili aracılarda hesap açarak işlemlerini bu hesaplar üzerinden yürütür. Bu model mevcut bankacılık sistemine benzer nitelikler taşır; ancak merkez bankası ile kullanıcı arasında doğrudan bir hukuki ilişki doğurması bakımından farklılık gösterir. Token tabanlı sistemlerde ise dijital para, fiziksel paraya benzer şekilde kullanıcı cüzdanlarında taşınır ve aktarılır; kimlik doğrulama yerine sahiplik esası ön plandadır. Bu model, anonimlik ve çevrimdışı kullanım gibi avantajlar sunsa da güvenlik, kayıp ve çifte harcama gibi sorunları beraberinde getirebilir.

CBDC mimarisinin bir diğer belirleyici unsuru da dağıtım modelidir. Doğrudan modelde merkez bankası, kullanıcıya doğrudan hizmet sunarken; iki katmanlı modelde özel bankalar veya yetkili ödeme sağlayıcılar aracı olarak işlev görür. Avrupa Merkez Bankası'nın önerdiği Dijital Euro modeli bu ikinci yapıyı esas almakta; böylece mevcut bankacılık sisteminin dışlanmaksızın dönüştürülmesini amaçlamaktadır.\footcite{ecb2023design} Ancak bu modelde özel sektör aktörlerinin üstlendiği rol, veri koruma, tüketici hakları ve şeffaflık bakımından yeni düzenleme ihtiyacını doğurmaktadır. Buna karşılık doğrudan modelde merkez bankasının son kullanıcıya karşı doğrudan yükümlülüğü oluşabilir; bu da kamu hukukunun devreye girdiği yeni bir sorumluluk alanı anlamına gelir.

CBDC’lerin sahip olabileceği programlanabilirlik özelliği, onları yalnızca bir ödeme aracı olmaktan çıkarıp, belirli koşullara bağlı olarak çalışan yazılım bileşenlerine dönüştürebilir. Bu tür bir yapı, sosyal yardım ödemelerinin yalnızca belirli tarihlerde veya belirli ürün gruplarında harcanması gibi uygulamaları mümkün kılmaktadır. Çin bu tür özellikleri pilot olarak e-CNY kapsamında test etmektedir. Öte yandan Avrupa Merkez Bankası, Dijital Euro'nun programlanabilir bir para birimi olmayacağını açıkça ifade etmiş, yalnızca programlanabilir ödemelere izin verileceğini vurgulamıştır. Bu yaklaşım tartışmalarıyla birlikte, “paranın tarafsızlığı” ilkesine dayandırılmakta ve bireysel özgürlükler ile piyasa nötrlüğü açısından programlanabilirliğin doğurabileceği riskleri en aza indirmeyi hedeflediği savunulmaktadır.\footcite{ecb2023digital}

CBDC mimarisine dair yapılacak her tercih, yalnızca teknik değil; aynı zamanda hukuki, anayasal ve ekonomik sonuçlar doğurmaktadır. Merkez bankalarının doğrudan veya dolaylı ilişki kuracağı kullanıcılar, bu sistemlerdeki veri yönetimi, sorumluluk dağılımı ve kamusal denetim ilkeleri bakımından yeni yorumlara ihtiyaç duyulmaktadır. Bu bağlamda, CBDC sistemleri yalnızca bir fintek (finansal teknoloji) ürünü değil; aynı zamanda yeni bir kamu hukuku paradigmasının adeta deney labaratuvarıdır.


\section{Akıllı Sözleşmeler ile Otomatik Uyuşmazlık Çözüm Modelleri}

Merkez bankası dijital paralarının programlanabilirliği, ödeme süreçlerinin otomasyonu ile sınırlı değildir; aynı zamanda sözleşmelerin ifası ve ortaya çıkabilecek uyuşmazlıkların çözümü bakımından radikal bir dönüşümü gündeme getirmektedir. Blokzincir teknolojisi üzerinde çalışan ve belirli koşulların gerçekleşmesi halinde kendiliğinden yürürlüğe giren akıllı sözleşmeler, klasik anlamda yargısal çözüm süreçlerine alternatif teşkil edecek bir mekanizma sunma potansiyeline sahiptir. Özellikle sözleşme ihlallerinin önlenmesi, ifa sürecinin gerçek zamanlı denetimi ve taraflar arasında güven gereksinimini ortadan kaldırması yönüyle bu yapılar, sözleşme hukukunun yeni bir evresine işaret etmektedir. Akıllı sözleşme sistemlerinin sağladığı deterministik yapı, bazı senaryolarda uyuşmazlık doğmadan çözüm üretmeyi mümkün kılmakta; bu da hukuk teorisinde uzun zamandır tartışılan “önleyici hukuk” (\textit{preventive law}) yaklaşımının teknoloji aracılığıyla somutlaşmasına yol açmaktadır.\footcite{lessig1999code}

Bununla birlikte, akıllı sözleşmelerin bu yapısal avantajı mutlak bir sorunsuzluk anlamına gelmemektedir. Her ne kadar kod üzerinden otomasyon sağlansa da, sistemin dış dünyaya olan bağımlılığı, yani dışsal veri kaynaklarına (\textit{oracle}) ihtiyaç duyması, beraberinde yeni hata riskleri doğurur. Akıllı sözleşme, kurgulanan koşulları doğru bir şekilde uygulasa bile, dışarıdan gelen veri hatalıysa sonuç da hukuken yanlış olacaktır. Örneğin bir teslimatın gerçekleşip gerçekleşmediğine dair bilgi, otomatik olarak bir taşıyıcı şirketin sisteminden alınmakta ve bu bilgiye dayanılarak ödeme işlemi yürütülmekteyse; söz konusu şirketin sisteminde yaşanacak teknik bir arıza, taraflardan birinin hak kaybına uğramasına neden olabilir. Bu durumda, kodun çalışması değil; verinin doğruluğu tartışma konusu olur. Dolayısıyla sistemin teknik deterministliği, hukuki gerçekliğe birebir denk düşmeyebilir. Bu da bizi klasik hukukta hâlâ geçerli olan takdir yetkisi, hakkaniyet değerlendirmesi ve insan müdahalesi gerekliliği gibi kavramlara geri götürür.

Bu teknik arka plana karşılık olarak blokzincir tabanlı “merkeziyetsiz uyuşmazlık çözüm sistemleri” geliştirilmeye başlanmıştır. En bilinen örneklerden biri olan \textit{Kleros}, tarafların önceden kabul ettiği şekilde, bir uyuşmazlık durumunda rastgele seçilen çevrimiçi jüri üyeleri tarafından oylama yapılmasını ve bu kararın zincir üzerinde otomatik olarak uygulanmasını sağlayan bir sistem sunmaktadır. Bu yapı, klasik tahkim sistemlerine alternatif olarak tasarlanmış olsa da henüz hukuken tanınan yargı organları kadar yetkili sayılmamaktadır.\footcite{wong2020kleros} Ancak sözleşmenin tarafları, aralarında açık bir irade beyanı ile bu tür dijital çözüm mekanizmalarını kabul etmişse, bu durum bazı hukuk sistemlerinde özel hakem kararı gibi değerlendirilerek bağlayıcılık kazanabilir. Türk hukuku açısından ise bu tür uygulamaların Borçlar Hukuku ilkeleriyle ne ölçüde örtüştüğü ve mahkemeler nezdinde geçerliliği halen tartışmaya açıktır.

CBDC uygulamalarında akıllı sözleşmeler yoluyla geliştirilen otomatik çözüm mekanizmaları yalnızca ticari sözleşmelerle sınırlı değildir; tüketici işlemleri de bu kapsamda düşünülebilir. Örneğin bir tüketici, CBDC kullanarak bir ürün sipariş etmiş ve belirli sürede teslimat gerçekleşmemişse, sistemin otomatik olarak iade işlemini başlatması mümkün hale gelir. Benzer şekilde, kira sözleşmelerinde ödeme yapılmadığında otomatik fesih işlemi veya gecikme tazminatının uygulanması senaryoları da teknik olarak hayata geçirilebilir. Ancak bu işlemlerde sistemsel hata, yanlış kodlama ya da kötü niyetli kullanım gibi ihtimaller devreye girdiğinde, haksız bir iade veya fesih işlemi meydana gelebilir. Bu tür durumlarda sorumluluğun hangi aktöre yükleneceği, nasıl düzeltileceği ve itiraz süreçlerinin hangi platformda yürütüleceği gibi hukuki sorular önem kazanır. Özellikle kamu gücünün kullanımı söz konusu olduğunda, yani idari para cezası tahsili, sosyal yardımın geri alınması veya vergi tahsilatı gibi konularda akıllı sözleşmelerin otomatik icra yeteneği, ciddi anayasal sorunları da beraberinde getirmektedir.\footcite{mas2023orchid}

Avrupa Merkez Bankası bu tür kaygılar ışığında, dijital euro sisteminin programlanabilir ödeme fonksiyonlarını destekleyeceğini ancak işlemlerin geri alınabilirliğini garanti altına alacak bir yapı öngördüğünü belirtmiştir.\footcite{ecb2023digital} Hukukun üstünlüğü ilkesi gereği, her bireyin işlem sonucunu yargı önüne taşıma hakkı vardır. Bu nedenle CBDC sistemleri içerisinde kullanılan akıllı sözleşmelerin, ne kadar otomatikleştirilmiş olursa olsun, manuel müdahaleye açık ve hukuki denetime elverişli yapılar olması gerektiği genel kabul görmektedir. Aksi halde, işlemlerin geri alınamazlığı ve kodun mutlaklığı prensibi, hukuk devleti ilkesine zarar verebilir.

Bu bağlamda ortaya çıkan çözüm önerisi, akıllı sözleşmelerin klasik yargı sistemini ikame etmesi değil; onu tamamlayıcı nitelikte bir “hibrit model” geliştirilmesidir. Otomasyonun sunduğu hız, maliyet avantajı ve tarafsızlık; hukuk sisteminin sunduğu takdir yetkisi, delil değerlendirmesi ve insan iradesiyle bütünleştirilmelidir. Akıllı sözleşmelerin CBDC uygulamalarında yer alması, ancak bu bütünlük içinde normatif meşruiyet kazanabilir. Bu çerçevede, bir sonraki bölümde Türkiye’de mevcut hukuk sisteminin bu tür otomatik yapılarla ne ölçüde uyumlu olduğu, hangi düzenleme boşluklarının bulunduğu ve nasıl bir hukuki çerçeve geliştirilmesi gerektiği ele alınacaktır.

\section{Türkiye’deki Hukuki Altyapı ve Mevzuat İhtiyaçları}

Merkez bankası dijital paralarının Türkiye’de uygulanabilirliği açısından en kritik meselelerden biri, mevcut hukuki çerçevenin bu yeni teknolojik yapılarla ne ölçüde uyumlu olduğudur. Dijital Türk Lirası gibi bir CBDC’nin akıllı sözleşmelerle entegre şekilde çalışacağı düşünüldüğünde, bu yapının sadece teknik altyapı değil, aynı zamanda güçlü ve açık bir normatif temele ihtiyaç duyduğu görülmektedir. Türkiye’de halihazırda yürürlükte olan düzenlemeler; ödeme sistemleri, elektronik para, bankacılık faaliyetleri ve kişisel verilerin korunması gibi alanlara odaklansa da, bunların hiçbiri doğrudan bir CBDC yapısına ve onunla birlikte gelen akıllı sözleşme teknolojisine hitap etmemektedir.

6493 sayılı Ödeme ve Menkul Kıymet Mutabakat Sistemleri Kanunu, elektronik para kavramını özel sektör eliyle çıkarılan, karşılığında fon alınan ve belli bir ödeme aracına yüklenen dijital değerler şeklinde tanımlamaktadır. Ancak bir CBDC, herhangi bir özel kuruluşun değil, doğrudan merkez bankasının ihraç ettiği, karşılıksız olarak sistemde değer olarak yer bulan, devlet güvenceli bir varlık olduğundan, bu tanım kapsamına girmemektedir. Bu durum, Dijital Türk Lirası'nın hukuki niteliğinin belirlenmesi açısından boşluk yaratmakta; kamu güvenceli dijital paraların hangi rejime tabi olacağı sorusunu gündeme taşımaktadır.\footcite{mas2023orchid} Aynı şekilde 1211 sayılı Türkiye Cumhuriyet Merkez Bankası Kanunu da, dijital para kavramına doğrudan yer vermemektedir. 1970 tarihli bu kanun, merkez bankasına para politikası ve banknot ihracı yetkisi vermekteyse de, dijital temsile dair hiçbir tanım veya yetki devrine rastlanmaz. Özellikle 52. maddede yalnızca fiziki banknotlara atıf yapılmakta; dijital paraların hukuken nasıl ihraç edileceği, hangi teknik ve kurumsal sınırlar içinde işleyeceği açık değildir. Dolayısıyla, TCMB'nin Dijital Türk Lirası Projesi kapsamında gerçekleştireceği faaliyetlerin meşruiyeti, ancak mevcut yasa hükümlerinin genişletici yorumlarıyla sağlanabilmektedir. Bu durumun ileride anayasal yetki tartışmalarına yol açmaması adına, ilgili yasal metinlerin dijital paraya özgülenmiş biçimde güncellenmesi zorunlu görünmektedir.\footcite{ecb2023digital}

Öte yandan, CBDC’nin dağıtımında iki katmanlı bir modelin benimsenmesi hâlinde, 5411 sayılı Bankacılık Kanunu da gündeme gelmektedir. Bu durumda merkez bankası, dijital parayı doğrudan bireylere değil; aracı finansal kuruluşlar vasıtasıyla dolaşıma sokacak ve bu bankalar, klasik mevduat toplayıcılığından farklı olarak, “dijital para arayüz sağlayıcısı” gibi yeni bir fonksiyon üstlenecektir. Bu rol, bankacılık hukuku açısından yeni sorumluluklar, denetim mekanizmaları ve lisanslama süreçleri gerektirebilir. Dolayısıyla BDDK’nın yetkileri ve denetim çerçevesi de CBDC’ye özgü şekilde yeniden yapılandırılmalıdır.\footcite{emmer2023cbdc} Ayrıca dağıtık sistemlerin doğası gereği, CBDC işlemlerinde yer alan tüm veri akışları, kişisel verilerin korunması açısından da 6698 sayılı Kanun kapsamında özel düzenlemelere ihtiyaç duyabilir. Dijital para ile yapılan her işlem, teknik olarak kullanıcının kimliğiyle eşleştirilebilir ve bu da bireylerin ekonomik mahremiyetinin devlet ya da aracı kurumlar tarafından izlenebilir hale gelmesi anlamına gelir. Bu noktada, veri anonimleştirme, açık rıza, veri minimizasyonu gibi prensiplerin CBDC sistemlerine nasıl entegre edileceği açık değildir. Özellikle geri dönülemezlik ilkesi ile veri silme hakkı gibi temel ilkeler arasında oluşabilecek çelişki, dikkatle ele alınmalıdır.

CBDC sistemleri yalnızca özel hukuk değil, kamu hukuku alanında da yeni düzenleme gereksinimlerini beraberinde getirmektedir. Özellikle akıllı sözleşmelerin kamu işlemlerinde kullanılması durumunda; örneğin vergi tahsilatının veya sosyal yardımların otomatik olarak gerçekleştirilmesi senaryolarında, idarenin işlem tesis etme yetkisi ile bireyin bu işleme karşı yargı yolu arama hakkı arasında yeni bir denge kurulması gerekecektir. Eğer işlem yalnızca kod vasıtasıyla tesis ediliyor ve herhangi bir beşeri karar süreci barındırmıyorsa, klasik anlamda idari işlem, karar ve itiraz mekanizmaları nasıl işletilecektir? Bu sorular, özellikle anayasa hukuku çerçevesinde “hukuki güvenlik” ve “hukuki öngörülebilirlik” ilkeleri bakımından değerlendirilmelidir. Akıllı sözleşmelerin sağladığı otomatiklik, yargı denetiminin dışında kalmamalı; aksine yeni yargısal denetim yöntemlerinin geliştirilmesini teşvik etmelidir.

Blokzincir teknolojisinin uluslararası niteliği ve sınır ötesi işlemlere açıklığı nedeniyle, uluslararası özel hukuk kapsamında da yeni düzenleme ihtiyacı doğmaktadır. Örneğin bir yabancı yatırımcının Türk CBDC altyapısı üzerinden işlem yapması halinde; işlemden doğan bir uyuşmazlıkta hangi ülkenin hukukunun uygulanacağı, yetkili mahkemenin neresi olacağı veya dijital bir tahkim kararının Türkiye'de tenfizi gibi meseleler, doğrudan 5718 sayılı Milletlerarası Özel Hukuk ve Usul Hukuku Hakkında Kanun’un kapsamına girer. Mevcut düzenlemeler dijital işlemleri ve yazılı olmayan irade beyanlarını sınırlı biçimde kapsadığından, bu alanda yorum ve uygulama farklılıklarının önüne geçmek adına açık normatif düzenlemelere ihtiyaç vardır.

Tüm bu gerekçeler çerçevesinde Türkiye’nin yalnızca mevcut mevzuatı yorumlamakla yetinmeyip, doğrudan CBDC’ye özgü yeni bir kanun yapması gerektiği görüşü ağır basmaktadır. Avrupa Birliği’nin Haziran 2023’te yayımladığı Dijital Euro yasa teklifi bu konuda örnek teşkil edebilir. Türkiye’de de TCMB, BDDK, KVKK, MASAK ve Adalet Bakanlığı gibi kurumların ortak katılımıyla, bütüncül bir yasa hazırlık süreci yürütülmesi yerinde olacaktır. Bu yasa, dijital paranın tanımından ihraç yetkisine, veri işleme kurallarından işlem geçerliliğine kadar pek çok alanda açık ve bağlayıcı hükümler içermeli; aynı zamanda yeni teknolojilerin hukukla barışık şekilde gelişmesini sağlayacak bir normatif altyapı sunmalıdır. Ancak bu şekilde hem dijital dönüşümün güvenle yönetilmesi hem de hukuk devleti ilkesinin teknolojik evrime karşı korunması mümkün olabilir.


\section{Uluslararası Karşılaştırmalı Hukuk: Avrupa Birliği, Çin, ABD, Singapur ve Hindistan}  

CBDC’ler konusunda ülkeler farklı yaklaşımlar benimsemekte olup, bu farklılıklar özellikle teknolojik mimari, programlanabilirlik seviyesi, veri gizliliği politikaları, merkeziyetsiz yapılarla entegrasyon ve akıllı sözleşmelere ilişkin hukuki çerçevede belirginleşmektedir. Bu bölümde, Avrupa Birliği (dijital euro), Çin (dijital yuan), Amerika Birleşik Devletleri, Singapur ve Hindistan örnekleri üzerinden 2024-2025 yıllarındaki güncel gelişmeler ışığında bir karşılaştırma yapılacaktır. Her alt bölümde, ilgili yargı alanının CBDC projesi ve akıllı sözleşmelerle entegrasyonu hukuki açıdan ele alınarak, düzenleyici çerçeve ve potansiyel etkiler değerlendirilecektir.

\subsection{Avrupa Birliği (Dijital Euro)}

Avrupa Birliği, dijital merkez bankası paraları konusundaki yaklaşımını temkinli fakat kararlı adımlarla geliştirmektedir. Euro Bölgesi’nde nakit kullanımının azalması, özel sektör temelli ödeme sistemlerine olan bağımlılığın artması ve dijital egemenliğin korunması gibi dinamikler, Avrupa Merkez Bankası (ECB) öncülüğünde yürütülen Dijital Euro projesini stratejik bir öncelik haline getirmiştir. Bu doğrultuda Avrupa Komisyonu, Haziran 2023’te dijital euronun yasal statüsünü düzenlemek üzere kapsamlı bir regülasyon teklifi sunmuş; teklif, dijital euronun fiziksel nakdin tamamlayıcısı olarak tanınmasını ve “kanuni para” statüsüyle tüm Euro Bölgesi’nde yaygın kabul görmesini öngörmüştür.\footcite{ecb2023digital} Tasarı, dijital euronun her birey ve işletme tarafından tam değeriyle kabul edilmesi gerektiğini vurgulamakla birlikte, mikro işletmelerin ve özel şahsi işlemlerin kapsam dışı tutulabileceği bazı esneklik alanları da bırakmaktadır.

Dijital euronun teknik ve hukuki tasarımı iki katmanlı bir mimariye dayanmaktadır. Merkez bankaları ihraç yetkisini elinde bulundururken, dağıtım ve müşteri hizmetleri özel sektör aracılığıyla sağlanacaktır. Bu bağlamda bankalar, elektronik para kuruluşları ve diğer ödeme hizmeti sağlayıcıları, dijital euro altyapısında aracı rol üstlenecektir. Banka hesabı olmayan bireylerin sisteme entegre edilebilmesi için ise posta ofisleri gibi alternatif kamu yapılarının devreye girmesi planlanmaktadır. ECB, sistemde finansal istikrarı korumak amacıyla dijital euro bakiyelerine üst sınır getirme yetkisini saklı tutmakta; böylece geleneksel mevduat sisteminin aşındırılmasının önüne geçilmesi hedeflenmektedir.

Programlanabilirlik açısından Avrupa Birliği son derece ölçülü bir yaklaşım benimsemektedir. ECB ve Komisyon, dijital euronun “programlanabilir bir para birimi” olmayacağını açıkça ifade etmiştir. Bu, dijital euro'nun doğrudan kendi içinde koşullandırılmış işlemleri desteklemeyeceği, yani kod seviyesinde belirli amaçlara tahsis edilmiş, tarih veya işlem türü gibi kriterlerle sınırlandırılmış bir yapıda tasarlanmayacağı anlamına gelir. Bununla birlikte, üçüncü taraf hizmet sağlayıcıların, dijital euro’yu kullanarak programlanabilir ödeme hizmetleri sunmasına hukuken alan tanınmaktadır. Bu noktada, dijital euro’nun kendisi değil, onun üzerinde çalışan uygulamalar programlanabilir nitelikte olacaktır.\footcite{ecb2023design} Söz konusu model, ödeme katmanının tarafsızlığını korurken, dijital inovasyonu destekleyen bir platform yaklaşımını öne çıkarmaktadır.

Veri mahremiyeti ve anonimlik konusu ise Avrupa Birliği açısından dijital euro tartışmalarının merkezinde yer almaktadır. Avrupa Veri Koruma Tüzüğü (GDPR) çerçevesinde şekillenen mevcut düzenlemeler, dijital ödemelerin bireysel mahremiyetle uyumlu olmasını zorunlu kılmaktadır. ECB, çevrimdışı işlemler yoluyla “nakde yakın gizlilik” sağlamayı planlamakta; bu bağlamda küçük tutarlı ödemelerde kimlik doğrulama gerektirmeyen bir sistem öngörmektedir. Ancak büyük ölçekli işlemler, kara para aklama ve terörün finansmanı ile mücadele ilkeleri gereği kayıt altına alınacak ve kullanıcı kimliğiyle eşleştirilecektir. Bu ikili yapı, mahremiyet ile AML (anti-money laundering) yükümlülükleri arasında denge kurmayı hedeflemektedir.

Avrupa Birliği'nin akıllı sözleşmeler konusundaki yaklaşımı da temkinlidir. ECB, dijital euro sisteminin merkezinde yer alacak bir programlanabilirlik yerine, sınırlı işlevli ve kontrollü akıllı sözleşme senaryolarını pilot testlerle değerlendirmiştir. Özellikle zincir üstü ödeme – zincir dışı teslim senaryolarında koşullu otomasyonun mümkün olduğu gösterilmiş; fakat henüz bu işlemlerin hukuki bağlayıcılığına dair kapsamlı bir düzenleme yapılmamıştır. Akıllı sözleşmelerin hukuki niteliği Avrupa Birliği hukukunda henüz açık biçimde tanımlanmamıştır. Ancak 2022 tarihli AB Veri Yasası (Data Act) taslağında, akıllı sözleşmelerin güvenli tasarım ilkeleri —özellikle değiştirilemezlik ve sonlandırılabilirlik— hakkında hükümler yer almaktadır. Bu durum, gelecekte akıllı sözleşmelere dair özel bir regülasyonun hazırlanabileceğine işaret etmektedir.

Özetlenecek olursa, Avrupa Birliği’nin Dijital Euro yaklaşımı ihtiyatlı ve kamu hukukuna dayalı bir model sunmaktadır. Programlanabilirlik sınırlı tutulmakta, mahremiyet en üst düzeyde korunmakta ve geleneksel hukuk prensiplerine uyumlu bir sistem inşa edilmeye çalışılmaktadır. ECB ve Avrupa Komisyonu'nun 2024-2025 yıllarında Dijital Euro’ya ilişkin hukuki çerçeveyi netleştirmesi ve uygulama rehberlerini yayımlaması beklenmektedir. Bu sürecin sonunda dijital euro’nun hem hukuk güvenliğine hem de teknolojik yeniliğe uyumlu bir model olarak şekillenmesi, Avrupa'nın dijital egemenliğini güçlendirme hedefiyle doğrudan ilişkilidir.


\subsection{Çin (Dijital Yuan – e-CNY)}

Çin Halk Cumhuriyeti, dijital merkez bankası paraları alanında küresel ölçekte en ileri perakende CBDC projelerinden birini yürüten ülkedir. Dijital Yuan (e-CNY) projesi, 2020 yılında başlatılan pilot program ile birlikte hızla yaygınlaştırılmış ve kısa sürede büyük şehirlerde uygulamaya alınmıştır. 2024 yılı itibariyle 26 büyük şehirde aktif biçimde kullanılan e-CNY, milyonlarca bireysel kullanıcıya ulaşmış, hem fiziksel mağazalarda hem de çevrimiçi platformlarda ödeme aracı olarak kabul görmüştür. Bu başarının temelinde Çin’in merkezi yapısı, teknolojik kapasitesi ve hızlı karar alma süreçleri bulunmaktadır. Hukuki açıdan ise Çin’in yaklaşımı, dijital parayı mevcut hukuki çerçeveye entegre etmekten ziyade, dijital yuan’ı yeni bir parasal form olarak tanımlayıp uygulamada idari düzenlemelerle kontrol altına almak şeklinde gelişmiştir.

e-CNY, Çin Halk Bankası (People’s Bank of China – PBoC) tarafından ihraç edilmekte ve iki katmanlı bir model çerçevesinde dağıtılmaktadır. Merkez bankası, dijital parayı ticari bankalara aktarırken; bu bankalar, kullanıcı cüzdanı açma ve ödeme hizmeti sunma gibi işlevlerle son kullanıcıya erişimi sağlar. Bu yapı, geleneksel bankacılık sisteminin dışlanmaksızın dijital paraya entegre edilmesini sağlamaktadır. 2024 yılı sonu itibariyle e-CNY ile gerçekleştirilen işlem hacmi 7 trilyon RMB'yi (yaklaşık 1 trilyon USD) aşmış; bu durum, Çin'in CBDC alanında küresel lider konumuna yükseldiğini göstermektedir.\footcite{pboc2021whitepaper}

Programlanabilirlik ve akıllı sözleşme entegrasyonu açısından Çin’in yaklaşımı oldukça yenilikçidir. Çin Halk Bankası, e-CNY'nin teknik altyapısına sınırlı da olsa akıllı sözleşme fonksiyonları entegre etmeye başlamıştır. 2023 yılı başlarında gerçekleştirilen bazı pilot uygulamalarda, e-ticaret platformları üzerinden gerçekleştirilen harcamalara bağlı olarak akıllı sözleşmelerin tetiklendiği görülmüştür. Örneğin, kullanıcılar belirli ürünleri satın aldıklarında alışveriş sepeti içeriğine göre sistem otomatik olarak ödül mekanizmalarını devreye sokmuş; bu da e-CNY'nin ticari kampanyalarla etkileşim kurabilen bir dijital para formuna evrildiğini göstermiştir. Ayrıca PBoC tarafından test edilen başka bir özellikte, belirli fonların yalnızca eğitim gibi sınırlı alanlarda kullanılabilmesi sağlanmış ve bu fonlar başka türde harcamalar için geçersiz kılınmıştır. Bu durum, merkez bankası paralarının yalnızca ekonomik değil, aynı zamanda davranışsal yönlendirme amacıyla da kullanılabileceğini ortaya koymaktadır.

Çin'in akıllı sözleşmelere yaklaşımında dikkat çeken bir başka unsur ise bu sistemlerin sıkı merkezi kontrol altında tutulmasıdır. PBoC’nin resmi açıklamaları ve dijital yuan teknik belgelerinde, sistemin “kontrollü anonimlik” ilkesine göre tasarlandığı belirtilmektedir. Bu ilke, küçük miktarlı işlemlerde bireylerin kimliğinin gizli tutulmasına izin verirken; büyük tutarlı, şüpheli ya da yasa dışı faaliyetlere ilişkin işlemlerde düzenleyici kurumların veriye erişimini mümkün kılar. Çin’in sosyal kredi sistemi ve genel finansal gözetim stratejileri düşünüldüğünde, e-CNY’nin mahremiyeti ikinci plana atan bir yapıda olduğu; buna karşılık devletin veri güvenliğini ve sistem bütünlüğünü öncelediği görülmektedir.\footcite{liu2022regulatory} Bu da Çin’in CBDC vizyonunun temelinde güvenlik, devlet kontrolü ve teknolojik egemenliğin yer aldığını göstermektedir.

Hukuki statü bakımından, Çin'de e-CNY açık biçimde yasal ödeme aracı olarak kabul edilmektedir. 2020 yılında Çin Halk Bankası Kanunu'nda yapılması planlanan değişiklik taslağında, dijital yuan’ın resmî para birimi olduğu ve hiçbir tüzel ya da gerçek kişinin bu parayı ödeme aracı olarak reddedemeyeceği hükmü yer almıştır. Ancak henüz bu değişiklik yasalaşmamış olup, PBoC dijital yuan'ı fiili düzenlemeler ve idari kararlarla uygulamaya sokmaktadır. Henüz özel bir “CBDC Yasası” çıkarılmamış olsa da, sistemin kapsamının büyümesi ve özel sektörle etkileşimlerinin artması, yakın gelecekte bu yönde normatif adımlar atılmasını zorunlu kılacaktır.

Siber güvenlik ve teknik standartlar da Çin’in CBDC yaklaşımında belirleyici bir rol oynamaktadır. e-CNY cüzdanları, çift çevrimdışı (dual offline) işlemleri destekleyecek şekilde tasarlanmış; yani hem alıcı hem gönderici çevrimdışıyken ödeme yapılabilmesine olanak tanınmıştır. Bu özellik, doğal afetler veya altyapı kesintileri gibi durumlarda da dijital para kullanımını mümkün kılmakta; dijital yuan’ı fiziksel nakde daha da yakınlaştırmaktadır. Akıllı sözleşmelerin ise yalnızca onaylanmış platformlar ve kontrollü koşullar altında çalıştırılmasına izin verildiği anlaşılmaktadır. Çin'in CBDC sistemini sıkı şekilde denetlemesi, bu teknolojilerin potansiyel risklerinin farkında olduğunu ve bu riskleri sistemik krizlere yol açmadan kontrol altına almak istediğini göstermektedir.

Yani genel olarak değerlendirildiğinde, Çin’in dijital yuan modeli, devlet merkezli ve yüksek gözetim temelli bir yaklaşıma dayanmaktadır. Bu modelde, programlanabilirlik unsuru etkin biçimde kullanılmakta; ancak bireysel mahremiyet, işlem anonimliği ve denetim dışı alanlar önemli ölçüde sınırlandırılmaktadır. Avrupa Birliği'nin daha liberal ve kullanıcı odaklı yaklaşımına karşılık Çin, dijital parayı kamu kontrol aracı olarak konumlandırmakta; bu da hukukun rolünü destekleyici değil, sınırlandırıcı biçimde yeniden tanımlamaktadır. Dolayısıyla Çin’in dijital yuan deneyimi, teknolojik açıdan öncü olsa da, bireysel haklar ve özgürlükler bakımından farklı yasal ve etik değerlendirmelere açık bir model sunmaktadır.


\subsection{���� Amerika Birleşik Devletleri}

Amerika Birleşik Devletleri, dijital merkez bankası paraları konusunda temkinli ve gözlemci bir yaklaşım benimsemektedir. Federal Rezerv Sistemi, bugüne dek herhangi bir dijital dolar (CBDC) uygulamasına geçilmesi yönünde resmî bir karar almamış; ancak konuyla ilgili kapsamlı araştırma ve değerlendirme süreçleri yürütmüştür. 2022 yılında yayımlanan “Money and Payments: The U.S. Dollar in the Age of Digital Transformation” başlıklı politika tartışma metni, bu yaklaşımın en somut örneğidir. Raporda CBDC’lerin ödeme sistemlerine olası katkıları, finansal kapsayıcılık, para politikası araçlarının genişletilmesi gibi potansiyel avantajları incelenmiş; öte yandan, bireysel mahremiyetin zayıflaması ve hükümetin bireylerin ekonomik faaliyetlerini doğrudan izleyebileceği bir sisteme dönüşme riski gibi endişelere de yer verilmiştir.\footcite{federalreserve2022money} Federal Rezerv, herhangi bir CBDC kararı alınmadan önce Kongre'nin yetkilendirmesi ve kamuoyunun desteğinin zorunlu olduğunu açık biçimde ifade etmiştir. Bu nedenle ABD’deki süreç, hâlen “araştırma ve politika tartışması” aşamasında bulunmaktadır.

ABD'nin CBDC'ye yaklaşımını belirleyen başlıca faktörlerden biri, doların halihazırda küresel rezerv para statüsüne sahip olması ve yerleşik dijital ödeme sistemlerinin oldukça gelişmiş olmasıdır. Venmo, Zelle gibi ödeme platformları ve kapsamlı kredi kartı altyapısı, dijital ödemelerin yaygınlaşmasını zaten sağlamış durumdadır. Bu nedenle CBDC, mevcut sistemin yerine geçecek bir zorunluluk olarak görülmemekte; daha ziyade küresel rekabette doların dijital egemenliğini sürdürmesinin bir aracı olarak değerlendirilmektedir. Ayrıca, ABD’de finansal teknolojilerin gelişimi büyük ölçüde özel sektör odaklıdır. Bu çerçevede özellikle “stablecoin” olarak bilinen özel sektör dolar token’ları (örneğin USDC, USDT) geniş bir kullanım alanına ulaşmış ve Kongre’de bu varlıklara yönelik yasal düzenleme çalışmaları başlamıştır. Ancak federal düzeyde henüz stablecoin’leri kapsayan bir yasa yürürlüğe girmemiştir.

Amerikan siyasetinde CBDC konusu aynı zamanda güçlü bir ideolojik tartışma alanı haline gelmiştir. Muhafazakâr kesimlerde, dijital bir doların hükümete bireylerin harcama alışkanlıklarını izleme ve müdahale etme imkânı sağlayabileceği yönünde ciddi kaygılar dile getirilmektedir. Bu bağlamda Temsilciler Meclisi üyesi Tom Emmer tarafından 2023 yılında sunulan “CBDC Anti-Surveillance State Act” adlı yasa tasarısı, Fed’in bireylere doğrudan CBDC ihraç etmesini yasaklamayı amaçlamıştır.\footcite{emmer2023cbdc} Tasarı, dijital paranın devlet eliyle programlanabilir bir gözetim aracına dönüşebileceği yönündeki endişeleri dile getirmiş ve bireysel mahremiyetin korunması gerekliliğine vurgu yapmıştır. Bu söylem, ABD’de CBDC tartışmasının yalnızca teknik değil; aynı zamanda anayasal haklar, ekonomik özgürlükler ve birey-devlet ilişkisi çerçevesinde şekillendiğini göstermektedir.

Teknik düzeyde ise ABD, dijital doların altyapı potansiyelini araştırmak üzere bazı pilot projeler gerçekleştirmiştir. Boston Federal Rezerv Bankası ile Massachusetts Institute of Technology (MIT) tarafından yürütülen Project Hamilton, dijital doların teknolojik uygulanabilirliğine dair bir prototip geliştirmiştir. Bu projede, saniyede yüzbinlerce işlemi gerçekleştirebilen bir dağıtık defter sisteminin test edildiği ve yüksek ölçeklenebilirlik ile gizlilik arasında denge kuran çözümlerin araştırıldığı belirtilmiştir.\footcite{hamilton2022project} Bunun yanında, New York Fed tarafından başlatılan Project Cedar kapsamında, toptan CBDC senaryoları ve sınır ötesi ödemeler üzerine testler yapılmıştır. Ancak bu projelerin hiçbiri doğrudan bir perakende dijital dolar ihraç kararı anlamına gelmemekte; yalnızca teknik bilgi birikiminin artırılmasını hedeflemektedir.

Federal düzeyde bir perakende CBDC’nin gündeme gelmesi halinde, mevcut yasal çerçevenin de gözden geçirilmesi gerekecektir. Mevcut Federal Reserve Act, Fed’e doğrudan tüketiciyle parasal ilişki kurma yetkisi tanımamaktadır. Dolayısıyla dijital doların bireylere sunulabilmesi için Kongre’den özel bir yasal yetkilendirme gerekecektir. Akıllı sözleşmelerin kullanımına dair ise herhangi bir federal düzenleme mevcut değildir. Şu anda yalnızca bazı büyük bankalar (örneğin JPMorgan) kendi dijital token’larını geliştirmekte ve sınırlı işlem türlerinde programlanabilir ödeme çözümleri test etmektedir. Bu uygulamalar ise doğrudan CBDC kapsamında değil, özel sektör dijital varlıkları kategorisinde değerlendirilmektedir.

Veri gizliliği konusu ABD’de oldukça hassas bir tartışma alanıdır. Ancak Avrupa Birliği’nin aksine, kapsamlı ve birleştirici bir federal veri koruma yasası bulunmamaktadır. Banka Gizliliği Yasası (BSA) gibi düzenlemeler daha çok kara para aklama ile mücadeleye yöneliktir ve bireysel mahremiyetin korunmasına dair net güvenceler sunmamaktadır. Bu nedenle, olası bir dijital dolar uygulamasında kişisel veri işleme süreçlerinin nasıl denetleneceği, anonimlik düzeyinin nasıl belirleneceği ve kullanıcıların hangi haklara sahip olacağı hâlâ belirsizliğini korumaktadır.

Toparlamak gerekirse ABD, CBDC konusunu henüz kurumsallaştırmamış; ancak çok boyutlu biçimde tartışan, teknik düzeyde hazırlık yapan ve siyasal meşruiyeti ön planda tutan bir model geliştirmektedir. ABD’nin özgürlükçü anayasal yapısı, bireysel özerkliğe duyarlılığı ve özel sektörün finansal inovasyonlardaki ağırlığı göz önüne alındığında, gelecekte oluşturulacak bir dijital dolar sisteminin minimal müdahale ilkesine göre tasarlanacağı söylenebilir. Programlanabilirliğin sınırlı, devlet müdahalesinin asgari düzeyde olduğu, özel sektör inovasyonuna açık ve mahremiyet öncelikli bir yapı, muhtemel bir ABD CBDC'sinin temel karakteri olabilir. Fakat yeni yönetimle birlikte ABD'nin tahmin edilemezliği arttığı için net bir yol haritası çizmek mümkün olamamaktadır.


\subsection{Singapur}

Singapur, dijital merkez bankası paraları konusundaki yaklaşımını yenilikçilik, esneklik ve düzenleyici öngörü çerçevesinde şekillendirmektedir. Hem bir finans merkezi hem de bir teknoloji laboratuvarı olarak konumlanan Singapur, CBDC araştırmalarını yalnızca teorik düzeyde bırakmayıp, sistemli pilot uygulamalarla zenginleştirmiştir. 2016–2020 yılları arasında yürütülen Project Ubin çalışmaları, Singapur Para Otoritesi’nin (Monetary Authority of Singapore – MAS) toptan kullanım amaçlı CBDC’ler üzerindeki ilk kapsamlı girişimi olarak öne çıkmıştır. Bu süreçte çok taraflı ödeme sistemleri, menkul kıymet takasları ve tokenleştirilmiş varlıkların değişimi gibi senaryolarda blokzincir teknolojisinin potansiyeli test edilmiş, proje sonunda dağıtık defter altyapılarının finansal piyasalarda işlevsel biçimde kullanılabileceği sonucuna varılmıştır.\footcite{mas2020ubin}

Perakende CBDC bağlamında ise Singapur, Project Orchid başlığı altında değerlendirme ve prototip geliştirme faaliyetlerine yönelmiştir. 2022 yılında yayımlanan ara raporda MAS, ülkenin mevcut elektronik ödeme altyapısının (PayNow, FAST gibi sistemler) oldukça etkin olduğunu, bu nedenle kısa vadede bir perakende CBDC’ye duyulan ihtiyacın sınırlı kaldığını ifade etmiştir. Buna karşın MAS, potansiyel bir dijital Singapur Doları için altyapı hazırlıklarını sürdürmektedir. Proje kapsamında, akıllı kart tabanlı CBDC prototipleri, çevrimdışı ödeme senaryoları ve programlanabilir işlem koşulları gibi teknik çözümlemeler gerçekleştirilmiş; bu da Singapur’un olası bir geçişi oldukça hazırlıklı şekilde yapabileceğini göstermektedir.

2024 yılına gelindiğinde Singapur’un odağı yeniden toptan CBDC uygulamalarına kaymıştır. Kasım 2023’te MAS, 2024 yılında gerçek işlemlerde kullanılacak toptan CBDC pilotlarına başlanacağını açıklamış ve bu kapsamda sınır ötesi ödemeler, teminat yönetimi, ticaret finansmanı gibi alanlarda dağıtık defter destekli çözümler üzerinde durulmuştur. Planlanan senaryolardan biri, farklı ülke CBDC’lerinin akıllı sözleşme temelli köprü protokolleri aracılığıyla anında takas edilmesidir. Bir diğer senaryo ise akıllı sözleşmelerle otomatikleştirilen teminat mektupları ve ticari ödeme taahhütleridir. Bu örnekler, Singapur’un yalnızca ulusal değil, aynı zamanda bölgesel ve küresel ölçekte dijital para sistemlerinin birlikte çalışabilirliğine odaklandığını ortaya koymaktadır.

Hukuki çerçeve açısından Singapur, esnek ve teknoloji dostu bir yaklaşımı benimsemektedir. CBDC’ye özgü bir yasal metin henüz bulunmamakla birlikte, mevcut mevzuat –özellikle 2019 tarihli Ödeme Hizmetleri Yasası– dijital ödeme sistemlerini düzenlemekte ve MAS’a bu alanda geniş düzenleyici takdir yetkisi tanımaktadır. Stablecoin’lere dair 2023 yılında yayımlanan rehber ilkeler, Singapur’un yenilikleri doğrudan yasaklamak yerine, bunlara nasıl hukuki zemin hazırlanabileceğini araştırdığını göstermektedir. Öte yandan, Singapur'un İngiliz ortak hukuk sisteminden gelen yapısı sayesinde, mahkemelerin teknolojiye hızlı adaptasyon gösterebildiği görülmektedir. Örneğin Quoine v. B2C2 (2020) davasında, bir kripto para borsasında akıllı sözleşme temelli işlemin iptali talebi incelenmiş ve tarafların sözleşmesel yükümlülükleri dijital ortam bağlamında yorumlanmıştır.\footcite{quoine2020case} Bu içtihat, Singapur hukuk sisteminin dağıtık uygulamaları da kapsayacak biçimde geliştiğini göstermektedir.

Programlanabilirlik bakımından Singapur, dijital paranın geleceğinde akıllı fonksiyonların temel bir yer tutacağına inanmaktadır. MAS Başkanı Ravi Menon, “akıllı para” kavramının finansal sistemde verimliliği ve otomasyonu artıracağını belirtmiş; bu doğrultuda akıllı sözleşme tabanlı senaryoların sandbox ortamlarında (regülasyon deneme ortamı) test edilmesine öncülük etmiştir. Yani merkez bankası veya regülatör, yeni teknolojiyi temkinli ama yenilikçi şekilde denemeye odaklanmıştır. Örneğin tahvil kuponlarının otomatik ödemesi, yatırım fonlarının koşula bağlı dağıtımı veya sosyal yardımların yalnızca belirli sektörlerde harcanabilmesi gibi senaryolar MAS tarafından desteklenen projeler arasında yer almıştır. Bu yaklaşım, CBDC’nin sadece değer transferi değil, aynı zamanda davranışsal yönlendirme ve finansal politika aracı olarak da kullanılabileceğini göstermektedir.

Veri gizliliği ve düzenleyici denetim konusunda Singapur dengeli bir duruş sergilemektedir. Avrupa Birliği’ndeki gibi katı bir veri koruma rejimi bulunmasa da, Kişisel Verilerin Korunması Yasası çerçevesinde finansal verilerin korunmasına dair asgari standartlar öngörülmektedir. CBDC uygulamalarında Singapur’un, tamamen anonim bir modelden ziyade, kimlik doğrulamalı fakat sınırlı veri erişimi sağlayan, izlenebilir fakat gizli bir model tercih etmesi beklenmektedir. Bu modelde teknik araçlarla (örneğin gizli işlemler, sıfır bilgi ispatı gibi) kullanıcı mahremiyeti korunurken; gerektiğinde denetleyici kurumların sisteme müdahalesine de olanak tanınacaktır.

Netice itibariyle Singapur, CBDC sistemleri açısından deneysel ama disiplinli bir yol izlemektedir. Mevcut dijital ödeme sistemlerinin yüksek verimliliği nedeniyle perakende CBDC’ye geçiş baskısı hissetmese de, hem teknik hazırlık hem de hukuki esneklik bakımından diğer ülkelere kıyasla oldukça avantajlı konumdadır. Türkiye gibi gelişmekte olan ülkeler açısından Singapur modeli, düzenleyici otoritenin proaktif davranarak pilotlar yoluyla bilgi üretmesi, özel sektörle güçlü iş birlikleri kurması ve akıllı sözleşmeleri sadece teknik değil, aynı zamanda hukuki bağlamda analiz etmesi bakımından örnek teşkil etmektedir.


\subsection{���� Hindistan}

Hindistan, dijital merkez bankası paraları alanında kısa sürede ciddi ilerleme kaydeden ülkelerden biri olarak öne çıkmaktadır. 2022 yılı sonunda Hindistan Merkez Bankası (Reserve Bank of India – RBI), hem perakende hem de toptan kullanıma yönelik olarak dijital rupi (e₹) pilotlarını başlatmış ve bu girişim, 2023–2024 yılları arasında kademeli olarak genişletilmiştir. Perakende segmentte pilot uygulama birçok şehirde yürütülmüş, dijital rupi çeşitli sektörlerde kullanım görmüş ve uygulamaya katılan banka sayısı artmıştır. 2024 yılı Mart ayı itibariyle, dijital rupi sisteminde kayıtlı bireysel kullanıcı sayısı 4.6 milyonu aşarken; 400 binden fazla ticari işletmenin de bu ödeme aracını kabul ettiği bildirilmiştir. Dolaşımdaki dijital rupi miktarı, bir yıl içinde 57 milyon rupi seviyesinden 2.34 milyar rupi seviyesine ulaşarak önemli bir artış göstermiştir.\footcite{rbi2023concept} Bu gelişmeler, Hindistan'ın CBDC’yi sadece deneysel değil, yaygınlaştırılabilir bir ödeme aracı olarak ele aldığını göstermektedir.

Dijital rupi tasarımı, Hindistan’ın mevcut finansal sistemine entegrasyon prensibi doğrultusunda yapılandırılmıştır. RBI, 1934 tarihli Merkez Bankası Kanunu'nda 2022 yılında yaptığı değişiklikle, dijital para ihraç yetkisini yasal zemine oturtmuş ve 2022 Mali Bütçesi’nde dijital rupi planlarını kamuoyuna açıklamıştır. RBI tarafından yayımlanan kavramsal çerçeve raporunda, dijital rupinin iki katmanlı bir modelle uygulanacağı; yani RBI’nin dijital parayı bankalar aracılığıyla son kullanıcılara ulaştıracağı belirtilmiştir. Perakende dijital rupi, token tabanlı ve cep telefonu destekli sistemler aracılığıyla işlev görmekte; cüzdanlar, mobil uygulamalar ve UPI altyapısıyla bütünleşik biçimde çalışmaktadır.

Hindistan’ın CBDC yaklaşımının en dikkat çekici unsurlarından biri, programlanabilirlik yeteneklerine verdiği önemdir. 2024 yılı itibariyle dijital rupiye belirli kullanım koşulları içerecek şekilde programlanabilir özellikler entegre edilmeye başlanmıştır. Özellikle yakıt, gıda, tarım, eğitim, sağlık ve ulaşım gibi sosyal öncelikli alanlarda, fonların belirli amaçlarla sınırlı kullanımına olanak tanıyan akıllı sözleşme benzeri sistemler denenmektedir. Nisan 2024’te bir özel bankanın gerçekleştirdiği örnekte, karbon kredisi karşılığında çiftçilere sağlanan dijital rupi ödemeleri, yalnızca gübre veya tohum alımında kullanılabilecek şekilde yapılandırılmıştır. Bu uygulama, CBDC’nin kamu destek mekanizmalarıyla entegrasyon potansiyelini göstermekte ve aynı zamanda kamu harcamalarının izlenebilirliğini artıran bir araç olarak işlev görmektedir.\footcite{bqprime2024carbon}

Hukuki açıdan bakıldığında, Hindistan CBDC uygulamalarını şimdilik genel mali mevzuat kapsamında yürütmekte; ancak programlanabilirlik ile birlikte ortaya çıkabilecek sorunlar için henüz özel bir düzenleme yapılmış değildir. Pilot süreçlerin RBI denetiminde yürütülmesi sayesinde temel işleyiş güvence altına alınsa da, akıllı sözleşme hatalarından doğabilecek zararlar, işlem iptalleri veya kullanıcı haklarının korunması gibi konular gelecekte normatif netlik gerektirecek hususlar olarak öne çıkmaktadır. Vergisel boyutta ise Hindistan Maliye Bakanlığı, dijital rupi işlemlerinin vergi rejimine entegre olduğunu açıkça beyan etmiş; bu tür işlemlerde gelir, KDV ve stopaj gibi yükümlülüklerin aynen geçerli olacağını ifade etmiştir. Bu yaklaşım, dijital rupi sisteminin kayıt dışı ekonomiyi azaltma ve kamu mali şeffaflığını artırma hedeflerine doğrudan hizmet etmesini amaçlamaktadır.

Kimlik doğrulama ve mahremiyet konularında Hindistan, diğer büyük ekonomilerden ayrılan bir yaklaşım sergilemektedir. Dijital rupi cüzdanları, ülkenin ulusal biyometrik kimlik sistemi Aadhaar ile doğrudan bağlantılıdır. Bu sayede dijital cüzdan açmak ve işlem yapmak için kimlik doğrulaması zorunludur. Uygulama, ayrıca cep telefonu numarasına dayalı çalıştığından, işlemler büyük ölçüde izlenebilir nitelik taşımaktadır. 2023 yılında yürürlüğe giren Dijital Kişisel Verilerin Korunması Yasası ile birlikte veri gizliliği konusunda genel bir yasal çerçeve oluşturulmuş olsa da, CBDC’ye özel detaylı düzenlemeler henüz yapılmamıştır. Buna karşın Hindistan devleti, dijital finansal sistemleri sübvansiyon politikaları ve vergi denetimiyle ilişkilendirmekte; bu nedenle tam anonimlikten ziyade izlenebilirlik odaklı bir model benimsemektedir.

Hindistan’ın CBDC yaklaşımı, hem kamu hem özel sektörle yürütülen iş birlikleri sayesinde şekillenen, destek odaklı ve kontrollü programlanabilirlik içeren bir “karma model”dir. RBI, mevcut yasalarını dijital rupi için uyarlamakla birlikte, daha geniş çaplı bir düzenleme ihtiyacının da farkındadır. Bu kapsamda “Cryptocurrency and Official Digital Currency Regulation Bill” adı altında bir yasa tasarısı zaman zaman gündeme gelmiş; ancak henüz yasalaşmamıştır. Yine de Hindistan, gelişmekte olan ülkeler açısından önemli bir örnek oluşturmaktadır: CBDC’yi teknolojik bir yenilikten ziyade ekonomik politika aracı olarak değerlendiren, aynı zamanda geniş nüfus gruplarına dijital erişim sunmayı amaçlayan bir yaklaşım benimsenmektedir. Bu da dijital para sistemlerinin yalnızca altyapısal değil; sosyal ve kamusal amaçlarla da şekillenebileceğini ortaya koymaktadır.









\begin{table}[h!]
\centering
\begin{tabular}{|l|l|l|}
\hline
\textbf{Ülke} & \textbf{Pilot Durumu} & \textbf{Hukuki Statü} \\
\hline
Avrupa Birliği & Pilot aşamasında & Yasal çerçeve hazırlanıyor \\
\hline
Çin & Geniş pilot & İdari düzenleme + yasa taslağı \\
\hline
ABD & Araştırma aşaması & CBDC için yasa yok \\
\hline
Singapur & İleri düzey toptan pilot & Genel mevzuat uyarlanıyor \\
\hline
Hindistan & Aktif perakende pilot & RBI Act değiştirildi, özel yasa bekleniyor \\
\hline
\end{tabular}
\caption{Seçili Ülkelerde CBDC Pilot Durumu ve Hukuki Statü}
\end{table}


\begin{table}[h!]
\centering
\begin{tabular}{|l|l|l|}
\hline
\textbf{Ülke} & \textbf{Programlanabilirlik} & \textbf{Anonimlik / Veri Gizliliği} \\
\hline
Avrupa Birliği & Sınırlı (3. taraf entegrasyon) & Nakde yakın çevrimdışı gizlilik \\
\hline
Çin & Orta-düzey, aktif kullanımda & Kontrollü anonimlik \\
\hline
ABD & İhtiyatlı, özel sektör testleri & Mahremiyet kaygısı ön planda \\
\hline
Singapur & Yüksek vizyon, sandbox odaklı & Kısmen gizli, denetim odaklı \\
\hline
Hindistan & Orta-düzey, kamu senaryoları & Açık kimlik doğrulama\\
\hline
\end{tabular}
\caption{Seçili Ülkelerde CBDC Programlanabilirlik ve Veri Gizliliği Özellikleri}
\end{table}







Verilen iki farklı karşılaştırma tablosu, Türkiye’nin de kendi yol haritasını çizerken bazı dengeleri gözetmesi gerektiğini ortaya koymaktadır. Avrupa gibi \textbf{kamu denetimi ve veri korumasına} ağırlık veren bir model ile Singapur gibi \textbf{teknolojik esnekliği} öne çıkaran bir model arasında, Türkiye için özgün bir denge bulunabilir. Nitekim bir değerlendirmeye göre, Türkiye CBDC konusunda bu iki modelin arasında konumlanabilir; programlanabilirliğe açık bir altyapı kurulurken, denetim mekanizmalarının da hukuka uygun şekilde tanımlanması gerekecektir.

\section{Senaryo Analizi: Dijital TL ile Kamu Alımında Akıllı Sözleşme Uyuşmazlığı}

Bu bölümde, \textbf{akıllı sözleşmelerin} (\textit{smart contracts}) ve \textbf{programlanabilir dijital para sistemlerinin} (\textit{programmable CBDCs}) kamu alımlarındaki potansiyel kullanımını konu alan kurgusal bir senaryo üzerinden teknik ve hukuki bir analiz yapılacaktır. Senaryo, \textbf{2027 yılında} Türkiye’de Dijital Türk Lirası’nın tüm kamu işlemlerinde aktif olarak kullanılmaya başlandığı bir dönemde geçmektedir.

Bir kamu kurumu olan Ulaştırma Bakanlığı, dijital devlet hizmetleri alanında faaliyet gösteren yerli bir teknoloji firmasıyla 12 milyon dolar bedelli bir sözleşme yapar. Sözleşmeye göre proje üç ayrı aşamada teslim edilecek ve her bir teslimatın ardından otomatik ödeme yapılacaktır. Bu işlem, Dijital TL altyapısında çalışan bir \textbf{akıllı sözleşme} ile yürütülecek; ödeme tetikleyicisi ise bir \textbf{oracle} (dış veri sağlayıcı) sistemi olacaktır. Her bir aşamanın “tamamlandığı” bilgisi sisteme girildiğinde, ilgili 4 milyon dolar tutarındaki ödemeler akıllı sözleşme tarafından otomatik olarak firmanın dijital cüzdanına aktarılacaktır.

İlk iki ödeme sorunsuz şekilde gerçekleştirilir. Ancak üçüncü aşamada, bakanlık temsilcisi sistemde yanlışlıkla “tamamlandı” bilgisini girer. Bu bilgi, oracle aracılığıyla akıllı sözleşmeye iletilir ve ödeme otomatik olarak yapılır. Oysa teslimat eksik yapılmıştır. Bakanlık, hatayı fark ederek firmanın ilgili tutarı iade etmesini ister; firma ise teknik şartlara göre ödeme yapılmış olduğunu ve yükümlülüğünü yerine getirdiğini ileri sürer.

Bu senaryo, \textbf{kodun hukuki işlemleri ikame etme kapasitesine} ve \textbf{yargı denetimi olmaksızın yapılan dijital işlemlerin} doğurabileceği sonuçlara dair kritik sorular ortaya çıkarmaktadır. Teknik açıdan bakıldığında, akıllı sözleşme kodu doğru işlemiştir; çünkü tasarlandığı kural dizisi tamamlanma verisini esas alır. Ancak bu veri hatalıdır ve gerçeklikten kopuktur. Blokzincir temelli sistemde işlem geri alınamaz olduğu için, ödeme değiştirilemez biçimde gerçekleştirilmiştir. Burada şuna dikkat çekmek gerekir. Blokzincir temelli sistemlerde işlem geri alınamazlığı esas olsa da önceden tasarlanmış mekanizmalarla bunun sağlanması imkansız değildir. Akıllı sözleşme geliştiricisi, sözleşmeye bilerek geri alma işlevi (örneğin revertTransaction, cancelTrade, refund) ekleyebilir. Bu durumda geri alma sadece önceden tanımlanan kurallar çerçevesinde mümkündür ve yetkili bir adres (örneğin yönetici ya da DAO) tarafından kullanılabilir. İkinci ihtimal "Yükseltilebilir \textit{Upgradeable}) Akıllı Sözleşmeler"dir. Proxy tabanlı sistemlerde, bir sözleşme mantık (\textit{logic}) sözleşmesiyle bağlantılıdır. Mantık sözleşmesi değiştirilerek sistemin işleyişi güncellenebilir. Böylece sözleşmenin davranışı değiştirilebilir ve dolaylı yoldan eski işlemler geçersiz kılınabilir. Çok nadir durumlarda (örneğin büyük saldırılar), blokzincir protokol düzeyinde bir hard fork yapılarak belirli işlemler geriye dönük olarak geçersiz kılınabilir. Bu, yukarıda da değinilen Ethereum tarihinde DAO krizinde bir kez gerçekleşmiştir. Bazı sözleşmelerde, bir işlem ancak birden fazla tarafın onayıyla kesinleşir. Bu onay tamamlanmadan işlem “geri çekilebilir”. Bu durumda işlem geri alınamaz değil, onaylanmamış sayılır. Bazı ikinci katman çözümler (\textit{L2}) (örneğin Optimistic Rollups), işlemleri bir süreliğine geçici olarak işler. Bu süre içinde itiraz gelirse işlem iptal edilebilir. Senaryoda bu beş alternatif mekanizmanın en başta kurulmadığı varsayılarak devam edilecek çünkü akıllı sözleşmelerin en güzel yanlarının başında bu müdehale edilemezlik karakteri gelmektedir. 

Hukuki açıdan mesele, kamu ihalesi ile bağlı idari bir sözleşmenin, teknik bir sistem aracılığıyla ifa edilip edilemeyeceğidir. Burada söz konusu olan şey, \textbf{eksik ifaya rağmen ödeme yapılmış olmasıdır}. Bu durum, klasik idare hukuku ilkelerine göre “sebepsiz zenginleşme”ye sebep olur. Ancak Türk hukuk sisteminde, idarenin hatayla yaptığı ödemelerin iadesi mümkündür ve bu tür durumlarda idare mahkemelerine başvurarak geri iade istenebilir. Firmanın iddiası ise teknik sistemin gereğini yaptığı ve dijital ortamda sözleşme hükmünün ifa edildiği yönündedir. Fakat bu savunma, mahkemeler nezdinde \textbf{“irade hatası”} (örneğin hatalı veri girişi) gibi durumlarla bertaraf edilebilir.

Bu senaryo, akıllı sözleşmelerde şu iki ilkeyi tartışmaya açmaktadır: \textbf{1)} Kod tarafından yapılan işlemler, hukuken her zaman geçerli midir? \textbf{2)} İnsan hatasına dayalı otomasyon sonuçları nasıl telafi edilir? Türk Borçlar Hukuku’nda irade sakatlığı, hile ve hata gibi durumlarda yapılan sözleşmelerin iptali mümkündür. Ancak burada dijital bir protokol, gerçek iradeyi yansıtmadan işlem yapmıştır. Bu nedenle dijital sistemin \textit{doğru işlem yapması}, her zaman \textit{hukuki olarak meşru olduğu} anlamına gelmemektedir.

Senaryodaki diğer önemli tartışma noktası ise, kamu hizmeti sunumunun \textbf{tamamen otomatik sistemlere devredilmesinin sınırlarıdır}. Eğer kamu idaresinin elinde bir “iptal” veya “durdurma” tuşu bulunmuyorsa, \textbf{anayasal sorumluluk} ve \textbf{hizmet kusuru} tartışmaları da gündeme gelecektir. Bu bağlamda, önerilen sistem tasarımında, her akıllı sözleşmeye bir \textbf{yargısal veya idari denetim mekanizmasıyla entegre edilebilecek durdurma fonksiyonu} eklenmesi tavsiye edilmektedir. Senaryoda ortaya çıkan uyuşmazlığın çözümü, teknik sistemin şeffaflığı ve delil niteliği açısından da önemlidir. Akıllı sözleşme kodunun işlevi, orakılın veri kaynağı, ödeme tetikleyicisi ve zaman damgası gibi unsurlar; mahkemede bilirkişiler tarafından incelenebilecek dijital deliller olarak değerlendirilmelidir. Bu tür sistemlerin, mevcut hukuki altyapıya entegre edilmesi için özel düzenlemeler gereklidir.

Buradan çıkarılabilecek temel sonuç şudur; \textbf{akıllı sözleşmelerin kamu hukuku süreçlerine entegre edilebilmesi için} yalnızca teknik yeterlilik değil, aynı zamanda yargı denetimi, hata düzeltme, irade sakatlığına karşı koruma ve geri alma hakları gibi hukuki ilkelerin de kodun içine gömülü olması gerektiğidir. “\textit{Code is law}” ilkesi, kamu işlemleri söz konusu olduğunda sınırlandırılmalı ve nihai karar merciinin insan ve hukuk olduğuna dair sistematik bir altyapı kurulmalıdır.


\section{Algoritmik Adalet ve Kodun Hukuki Niteliği}

Merkez bankası dijital paraları (CBDC) ile birlikte kamu gücünün yazılım sistemleri üzerinden icrası gündeme gelmiş; \textbf{akıllı sözleşmeler} (\textit{smart contracts}) gibi algoritmik yapılar, yalnızca teknik bir araç olmaktan çıkıp doğrudan hukuki sonuç doğuran mekanizmalar hâline gelmiştir. Bu dönüşüm, hukuk sisteminin temel ilkeleri ile teknoloji arasındaki ilişkiyi yeniden düşünmeyi zorunlu kılmaktadır. “\textbf{Hukuk kuralı}” ile “\textbf{kod}” arasında var olan sınır bulanıklaşmakta; \textit{otomatik işlemler}, \textit{yazılımın bağlayıcılığı} ve \textit{denetlenebilirlik} gibi kavramlar hukuk kuramının merkezine yerleşmektedir.

\subsection{“Kod Hukuk Olur mu?”: Normatiflik Tartışması}

Dijitalleşmenin erken dönemlerinden itibaren yazılımın davranışları düzenleme biçimi, hukukçuların dikkatini çekmiştir. Bu alandaki en etkili tez, \textbf{Lawrence Lessig} tarafından 1999 yılında ortaya atılmıştır. Lessig, “\textit{Code is Law}” adlı eserinde, dijital ortamlarda bireylerin davranışlarını yönlendiren temel gücün yalnızca hukuk kuralları değil, aynı zamanda \textbf{yazılım kodu} (\textit{source code}) olduğunu belirtmiştir. Ona göre bir bireyin hangi siteye erişebileceği, ne tür içerikler üretebileceği, nasıl bir finansal işlem gerçekleştirebileceği; bu işlemlerin nasıl ve kim tarafından kaydedileceği doğrudan kod tarafından belirlenmektedir. Böylece kod, klasik anlamda yasa koyucunun yerine geçebilmektedir. Bu görüş, blokzincir teknolojilerinin gelişimiyle daha da somut bir boyut kazanmıştır. Özellikle \textbf{akıllı sözleşmelerin}, dış müdahaleye kapalı şekilde işlem gerçekleştirmesi; yazılımın, sözleşmesel ilişkiyi bizzat icra eden bir özne gibi çalışmasına neden olmuştur. Bunun en dramatik örneklerinden biri, 2016’da Ethereum ağında yaşanan “\textit{The DAO}” olayında görülmüştür. Burada kodda bulunan bir açığın kötüye kullanılması sonucu milyonlarca dolarlık varlık başka bir hesaba aktarılmış; topluluk ikiye bölünmüş ve etik olarak “kodun söylediği geçerlidir” anlayışı ile “amaç ve hakkaniyet esastır” anlayışı karşı karşıya gelmiştir.

Ancak “\textit{code is law}” anlayışı ciddi eleştiriler de almıştır. Kodun \textbf{normatiflik} niteliği tartışmalıdır; çünkü kod, demokratik bir sürecin ürünü değildir. Hukuk, kamuya açık süreçler, istişareler ve denge-denetim mekanizmalarıyla oluşurken; kodlar çoğu zaman yazılımcılar veya özel şirketlerce üretilmekte ve kamu denetimine açık olmamaktadır. Bu nedenle bazı yazarlar, \textit{“code regulates, but it is not law”} (kod düzenler, ama hukuk değildir) yaklaşımını savunarak yazılımın hukuki normatifliğinin sınırlı olduğunu ileri sürmektedir.

\subsection{Algoritmik Adalet ve Hukuk Devleti İlkesi}

“Kodun düzenleyici gücü” tartışması, doğrudan \textbf{algoritmik adalet} (\textit{algorithmic justice}) kavramını gündeme getirmiştir. Bu kavram, algoritmaların doğuracağı sonuçların \textbf{hukuka, hakkaniyete ve temel haklara uygunluğunu} sorgulayan bir düşünce çerçevesidir. Eğer yazılım bir kararı otomatik olarak veriyor ve insan müdahalesi olmadan sonucu doğuruyorsa, o zaman bu kararın denetlenebilirliği, düzeltilebilirliği ve hakkaniyete uygunluğu da garanti altına alınmalıdır. Örneğin bir \textbf{CBDC} sisteminde, bir vatandaşın sosyal yardım hakkı otomatik olarak reddedilirse veya vergi tahsilatı herhangi bir yargı süreci olmaksızın gerçekleşirse, bu işlemler klasik hukuk düzenindeki \textbf{“yargı güvencesi”} ilkesi ile çatışacaktır. Oysa hukuk devleti, bireyin haklarının \textit{önceden belirli, şeffaf ve denetlenebilir} kurallarla güvence altına alınmasını şart koşar. Akıllı sözleşmeler bu anlamda potansiyel bir tehdit olduğu kadar, doğru tasarlandığında adaletin daha etkin sağlanması için bir araç da olabilir.

Bu bağlamda, algoritmik işlemlerin hukukla uyumlu olabilmesi için bazı ön koşullar önerilmektedir:

\begin{itemize}
  \item Kodların şeffaf ve denetlenebilir olması
  \item Geri alınabilirlik ve iptal mekanizmalarının bulunması
  \item İnsan müdahalesine açık ara yüzlerin sağlanması
  \item Hatalı kararlar için etkili bir yargı yolu sunulması
  \item Kodun kendisinin de normatif çerçeve içinde değerlendirilmesi
\end{itemize}

CBDC sistemlerinde ve genel olarak dijital altyapılarda kullanılan kodlar, yalnızca teknik araçlar değil; aynı zamanda davranışı belirleyen normatif çerçeveler haline gelmektedir. Ancak bu normatiflik, klasik hukukun yerini alamaz. Kodun \textbf{kendisi} de hukuk tarafından şekillendirilmek zorundadır. Aksi takdirde, dijital çağda bireyin hak ve özgürlükleri özel sektörün veya teknik sistemlerin insafına terk edilmiş olur. Bu nedenle, “kodun hukuk olması” değil; “\textit{hukukun kodu yönlendirmesi}” gereklidir. Algoritmik adalet fikri, bu dengeyi kurmaya yönelik çağdaş bir hukuk arayışıdır. \textbf{Algoritmik adalet} (\textit{algorithmic justice}), dijital sistemlerin, özellikle de \textbf{otomatik karar alma mekanizmalarının} (\textit{automated decision-making systems}) bireyler ve gruplar üzerinde eşit, adil ve şeffaf etkiler yaratmasını amaçlayan normatif bir çerçevedir. Bu kavram, yalnızca teknik doğrulukla yetinmeyen; aynı zamanda toplumsal, hukuki ve etik ölçütleri de gözeten bir denetim anlayışını ifade eder. Giderek artan biçimde hukukî işlemler, kamu hizmetleri, finansal kararlar ve vatandaş-devlet etkileşimleri algoritmalar aracılığıyla yürütülmektedir. Bu bağlamda algoritmalar yalnızca veri işleyen yapılar değil, aynı zamanda \textbf{normatif sonuçlar} doğuran düzenleyicilerdir.

\textbf{CBDC sistemlerinde} algoritmik adaletin önemi daha da artmaktadır. Çünkü merkez bankası dijital paraları, hem kamusal hem bireysel düzeyde çok sayıda işlem ve karar sürecini otomatikleştirir. Özellikle \textbf{akıllı sözleşmeler} ve \textbf{programlanabilir ödeme sistemleri}, belirli koşullar gerçekleştiğinde kendiliğinden sonuç doğurur. Bu işlemlerde adaletin sağlanabilmesi için, sistemin yalnızca hatasız çalışması değil; aynı zamanda \textit{hukukî denetim}, \textit{itiraz hakkı} ve \textit{ölçülülük} gibi ilkelere uygun çalışması da gerekir. Örneğin bir dijital yardım ödemesinin yalnızca belirli ürünlerle sınırlı olarak programlanması, teknik olarak mümkün olsa da, bireyin tüketim özgürlüğünü aşırı sınırlayarak \textbf{ölçülülük ilkesini} ihlal edebilir. Ya da sistem hatalı bir şekilde vatandaşın dijital cüzdanını dondurduğunda, bu işlemin geri alınamaz olması \textbf{mülkiyet hakkı} ve \textbf{yargı yoluna erişim hakkı} açısından ciddi sorunlar yaratabilir.

Algoritmik adaletin temel ilkeleri literatürde dört başlık altında özetlenmektedir:

İlk olarak, \textbf{şeffaflık} (\textit{transparency}) ilkesine göre, otomatik sistemlerin hangi verilerle ve hangi kurallara göre işlem yaptığı açıklanabilir olmalıdır. Bir akıllı sözleşme belirli bir transferi reddediyorsa, bunun nedeni kullanıcıya açık biçimde bildirilmeli ve algoritmanın mantığı denetlenebilir olmalıdır.

İkinci olarak, \textbf{izlenebilirlik} (\textit{traceability}) sistemi, her kararın hangi tetikleyici olayla alındığını, hangi veri akışıyla işlendiğini, hangi kod bloğu tarafından uygulandığını ortaya koyabilmelidir. Blokzincir sistemlerinin sunduğu değiştirilemez kayıt özelliği bu noktada faydalı olabilir; ancak kayıtlar şeffaf değilse, bu izlenebilirlik anlamsız hâle gelir.

Üçüncü olarak, \textbf{itiraz hakkı} (\textit{right to contest}) anayasal hukuk düzeninde temel bir ilkedir. Bir bireyin kendisine yapılan bir dijital işlem veya karar karşısında, bir üst denetim merciine başvurabilmesi gerekir. Bu, yalnızca mahkemeler değil; aynı zamanda idari denetim mekanizmaları yoluyla da sağlanabilir. Örneğin, TCMB'nin otomatik olarak gerçekleştirdiği bir para blokajına karşı bireyin itiraz edebileceği bir bağımsız dijital ombudsman mekanizması geliştirilebilir.

Son olarak, \textbf{hukuki düzeltilebilirlik} (\textit{legal rectifiability}) ilkesi gereğince, bir kod parçası ya da algoritmanın ürettiği sonuç, eğer üst hukuk normlarına aykırıysa, iptal edilebilmeli veya telafi edilebilmelidir. Hukuk hiyerarşisinin dijital ortama da uygulanabilmesi bu ilkenin temelidir. Aksi halde yazılım tarafından yaratılan durumlar anayasal denetim dışı kalacaktır ki bu, hukuk devleti ilkesiyle bağdaşmaz.

Türkiye Cumhuriyeti Anayasası’nın 2. maddesinde belirtilen \textbf{hukuk devleti} ilkesi, bu tartışmaların merkezinde yer almaktadır. Bu ilke, yalnızca kuralların varlığını değil; aynı zamanda bu kuralların erişilebilir, öngörülebilir ve denetlenebilir olmasını da içerir. Tam otomatik ve hataya kapalı olmayan akıllı sistemlerde, özellikle \textbf{idari hata}, \textbf{insan müdahalesi} ve \textbf{veri güvenliği} sorunları göz önünde bulundurularak algoritmik adalet çerçevesinde çözümler geliştirilmelidir. CBDC özelinde bu, TCMB’nin yalnızca teknik bir platform değil, aynı zamanda \textbf{hukukî olarak denetlenebilir bir altyapı} kurması anlamına gelir. Dijital TL sistemi içinde geliştirilecek her tür akıllı sözleşme, algoritmik adalet ilkelerine uygun bir \textit{yazılım anayasası} çerçevesinde değerlendirilmelidir.


\subsection{�� Anayasal Perspektiften Değerlendirme}

\textbf{Hukuk devleti} (\textit{rule of law}) ilkesi, Türkiye Cumhuriyeti Anayasası’nın 2. maddesinde yer alan temel kurucu değerlerden biridir. Bu ilke, devletin tüm işlem ve eylemlerinin hukuka uygun olmasını, yargı denetimine açık olmasını ve birey haklarının güvence altına alınmasını zorunlu kılar. \textbf{Akıllı sözleşmeler} (\textit{smart contracts}) ve \textbf{programlanabilir para sistemleri} gibi teknolojik araçların kamu otoriteleri tarafından kullanımı ise bu temel ilkenin yeni dijital bağlamda yeniden yorumlanmasını gerektirmektedir.\autocite{akdeniz2021yapay}

Öncelikle, Anayasa'nın 125. maddesi uyarınca, idarenin her türlü işlem ve eylemi yargı denetimine tabidir. Dolayısıyla, bir kamu kurumunun akıllı sözleşme aracılığıyla gerçekleştirdiği işlem, yalnızca teknik bir kod çalıştırması gibi görülse bile, hukuken denetlenebilir nitelikte olmalıdır. Aksi durumda, idarenin keyfîliğini sınırlandıran temel ilke olan \textbf{yargısal denetim} mekanizması işlevsiz kalır.\autocite{kaya2020anayasaya}

İkinci olarak, akıllı sözleşme temelli sistemlerde \textbf{temel hak ve özgürlüklerin} korunması kritik önemdedir. Özellikle \textbf{mülkiyet hakkı}, \textbf{özel hayatın gizliliği}, \textbf{kişisel verilerin korunması} ve \textbf{hak arama özgürlüğü} gibi anayasal güvenceler, dijital ortamlarda da aynen geçerlidir. Örneğin, CBDC sisteminde bir bireyin dijital cüzdanına hukuksuz biçimde bloke konulması, Anayasa'nın 35. maddesi kapsamında mülkiyet hakkını ihlal edebilir. Benzer şekilde, işlem geçmişi ve harcama profili gibi finansal verilerin aşırı biçimde izlenmesi, 20. madde çerçevesinde özel hayatın gizliliği ilkesine aykırılık teşkil edebilir. Bu nedenle, \textbf{veri minimizasyonu}, \textbf{orantılılık} ve \textbf{amaçla sınırlı kullanım} gibi prensipler, CBDC sistem tasarımı için anayasal ölçütler hâline gelmektedir.\autocite{turkel2022gizlilik}

Üçüncü olarak, \textbf{normlar hiyerarşisi} çerçevesinde kodun konumu değerlendirilmelidir. Kod, bir düzenleme biçimi olsa da hukuki bir norm değildir. Kanunlar ve Anayasa’ya aykırı bir kod parçasının teknik olarak uygulanabilir olması, onun hukuken geçerli olduğu anlamına gelmez. Özellikle kamu otoritelerince kullanılan akıllı sözleşmelerde bu husus daha da önem kazanır. Kodun herhangi bir şekilde ayrımcılığa, eşitlik ilkesinin ihlaline veya açıkça yasal yetki dışında sonuçlara yol açması durumunda, bu kodun \textbf{hukuka aykırılığı} kabul edilmelidir. Hukuki normlara uygunluk denetimi yalnızca içerik açısından değil, teknik tasarım açısından da sağlanmalıdır. Bu, hukukçular ile yazılım geliştiricilerin \textbf{disiplinlerarası iş birliğini} zorunlu kılar.\autocite{brownsword2020lawtech}

Anayasa’nın 73. maddesi çerçevesinde ayrıca \textbf{kanunilik ilkesi} göz önünde bulundurulmalıdır. Vergi ve benzeri yükümlülükler, yalnızca kanunla konulabilir. Oysa bazı programlanabilir ödeme sistemleri, belirli davranışlara yaptırım uygulayan, veri temelli analizlerle sonuç doğuran algoritmalar içerdiğinde, fiilen kural koyucu gibi işlev görebilir. Bu durumda, yürütmenin yasama yetkisini aşacak şekilde düzenleme yapması riski ortaya çıkar. Bu da yasama yetkisinin devredilemezliği ilkesiyle çelişir.

Tüm bu anayasal ilkeler ışığında, CBDC ve akıllı sözleşme sistemleri geliştirilirken idarenin algoritmik kararlarının yargı denetimine açık olması, temel hakların dijital ortamlarda da güvence altına alınması, akıllı sözleşme kodlarının üst normlara uygunluk açısından hukukçularca ön denetime tabi tutulması ve kod ile doğan hukuki işlemlerin geriye dönük düzeltmeye açık olması ile bireylerin itiraz hakkının garanti altına alınması gerekmektedir. Bu bağlamda, Türkiye özelinde Türkiye Cumhuriyet Merkez Bankası (TCMB) başta olmak üzere düzenleyici kurumların, CBDC sistemini anayasal ilkeler doğrultusunda yapılandırması zorunludur; aksi takdirde, birey haklarını zedeleyen teknik kararlar hukuk devleti ilkesinin zayıflamasına yol açabilir. Anayasa’nın öngördüğü normatif yapı, dijital dönüşüm çağında da geçerliliğini korumalıdır.


\subsection{Önerilen Denge: Hibrit Normatif Sistem}

Akıllı sözleşmelerin kamu işleyişine entegre edilmesi, yalnızca teknik bir otomasyon değil; aynı zamanda hukuki bir yeniden yapılanma sürecidir. Bu bağlamda, teknik işlem süreçleri ile normatif düzen arasındaki dengeyi sağlamak üzere önerilen yaklaşımlardan biri, literatürde ``\textbf{hibrit normatif sistem}''  olarak adlandırılmaktadır. Bu sistemde ne kod mutlak otorite olur, ne de insan müdahalesi tüm kararları belirler; bunun yerine kod ve hukuk birlikte çalışır.

\textbf{Hibrit yapının ilk ve temel prensibi}, kodun yalnızca işlemsel (teknik) görevleri yerine getirmesi, normatif kararlarda ise son sözün hukuka ait olmasıdır. Örneğin, ödeme limiti kontrolü, işlem zamanı denetimi gibi teknik adımlar kod tarafından yürütülürken; sözleşmenin geçerliliği, işlem koşullarının adilliği gibi noktalar hukuki değerlendirmenin alanına bırakılır. Bu, algoritmaların hız avantajını korurken, hukuki değerlendirme gerektiren alanlarda insan müdahalesini mümkün kılar.

İkinci temel prensip, \textbf{kodun güncellenebilir ve normlara uyumlu biçimde tasarlanmasıdır}. Akıllı sözleşmeler yalnızca bir kere yazılıp zincire gömülen statik yapılardan ibaret olmamalıdır. Aksine, sistem tasarımında, kanun değişikliklerine tepki verebilecek, gerektiğinde kodu durdurabilecek veya geçici olarak askıya alabilecek mekanizmalar yer almalıdır. Özellikle kamu işlemlerinde, kodların normatif değişikliklere duyarlı olması, hukuk devleti ilkesiyle uyumu açısından hayati önem taşımaktadır.

Üçüncü ve belki de en kritik prensip, \textbf{kodların yargısal denetime açık olmasıdır}. Hukuk devleti, işlemlerin yargı denetimi altına alınmasını zorunlu kılar. Bu bağlamda, teknik işlemler de sonuçları itibariyle yargı mercilerince gözden geçirilebilmeli; örneğin bir kodun işlettiği ödeme, hatalıysa iptal ettirilebilmeli veya geri alınabilmelidir. Bunun teknik zorlukları olsa da, özellikle kamuya ait sistemlerde bu tür ``geri alma'' ya da ``geçersizlik'' mekanizmalarının açıkça tanımlanması gerekir.

Bu tür bir hibrit yapı, uygulamada çeşitli teknik ve kurumsal düzenlemelerle somutlaştırılabilir. Örneğin, Türkiye Cumhuriyet Merkez Bankası (TCMB) bünyesinde oluşturulacak bir \textbf{Dijital TL Denetleme Komitesi}, akıllı sözleşmelerin çalışmasını izleyebilir, kamu şikayetlerini değerlendirebilir ve gerektiğinde teknik müdahale kararı alabilir. Böylece sistem kendi içinde otonom çalışırken, hukukî denetim ve müdahale için bir ``bekçi mekanizması'' hazır bulunur. Bu yaklaşım, \textit{law-by-design} olarak bilinen ve hukukî değerlerin sistem tasarımına baştan dahil edilmesini savunan teorik yaklaşımla örtüşmektedir. Özellikle Prof. Mireille Hildebrandt ve Roger Brownsword gibi hukukçular, teknolojik sistemlerin yalnızca teknik değil, aynı zamanda normatif varlıklar olduğunu vurgulayarak; tasarım aşamasında hukukçuların rolünü kaçınılmaz görmektedir.

Bütüncül bir bakışla değerlendirildiğinde, hibrit normatif sistem yaklaşımı, CBDC sistemleri gibi hem teknik hem kamu hukuku niteliği taşıyan platformlar için ideal bir denge modelidir. Bu modelde hız, verimlilik ve otomasyon korunurken; denetlenebilirlik, hukuki esneklik ve temel haklara saygı da garanti altına alınabilir. Türkiye’nin dijital dönüşüm sürecinde bu yaklaşımı benimsemesi, hukuk devleti ilkesinin dijital çağda da etkinliğini sürdürmesi açısından kritik olacaktır.


\section{�� Sonuç ve Politika Önerileri}

Merkez bankası dijital paralarının (CBDC) finansal sistemler üzerinde yaratacağı dönüşüm, yalnızca teknik bir yenilik olmayıp, aynı zamanda hukukun işleyişini ve devletin normatif yapısını da yeniden şekillendirme potansiyeline sahiptir. Akıllı sözleşmelerle desteklenen CBDC sistemlerinin yaygınlaşması, klasik hukuk ilkeleriyle çelişmeden nasıl yönetileceği sorusunu gündeme getirmiştir. Bu makalede yapılan analizler ve karşılaştırmalar ışığında, aşağıdaki tespitler ortaya konulmuştur:

\begin{itemize}
  \item \textbf{CBDC sistemleri}, ödeme işlemlerini hızlandırmakta ve kayıt dışı ekonomiyi azaltma potansiyeli taşımaktadır. Dijital ortamda para hareketlerinin şeffaflaşması, vergisel uyumu ve mali disiplini artırabilir. (Örneğin, Hindistan dijital rupisi pilotu nakitsiz ödemeleri belirgin şekilde artırmıştır.)
  \item Ancak, akıllı sözleşmelerle yapılan \textbf{otomatik işlemler}, geri döndürülemezlik ve denetimsizlik gibi riskler barındırmaktadır. Kod hataları veya öngörülmeyen senaryolar, taraflar için hak kayıplarına yol açabilir. (Senaryo analizimizde Bakanlığın yaşadığı sorun bu riski göstermiştir.)
  \item \textbf{Mevcut Türk mevzuatı}, blokzincir tabanlı ve otomasyon içeren bu tür sistemlerin doğasına uygun değildir. Mevzuatımız, elektronik parayı özel kuruluşlar çerçevesinde ele almakta ve merkezi denetime göre şekillenmiştir. Ayrıca akıllı sözleşmelerin sözleşme hukuku ve idare hukuku açısından doğuracağı meseleler düzenlenmemiştir.
  \item Uluslararası karşılaştırmalar, farklı ülkelerin \textbf{kontrollü programlanabilirlik} ve \textbf{denetimli otomasyon} modellerini benimsediğini göstermektedir. Örneğin AB, sınırlı programlanabilirliği seçerken; Çin sıkı devlet kontrolüyle ama bazı akıllı sözleşme özellikleriyle ilerlemektedir.
  \item Senaryo analizleri, özellikle kamu işlemlerinde \textbf{esnek müdahale} ve \textbf{hukuk devleti ilkesinin korunması} gerekliliğini vurgulamaktadır.
\end{itemize}


Dijital Türk Lirası’na özel bir yasal çerçevenin oluşturulması kritik olup, “Dijital Türk Lirası Kanunu” başlığıyla çıkarılacak bir yasa, dijital paranın tanımını, TCMB’nin yetki ve sorumluluklarını, dijital para ile ilgili hak ve yükümlülükleri ayrıntılı şekilde düzenlemelidir. Akıllı sözleşmelerin hukuki güvenliği açısından, belirli idari birimlerin gerektiğinde müdahale edebileceği denetlenebilir “override” mekanizmalarının tanımlanması önem arz etmektedir. Ayrıca, algoritmik kararların yargı denetimine açık olması sağlanmalı, bu bağlamda dijital sistemlerden kaynaklanan uyuşmazlıkları inceleyecek yargıç ve bilirkişilerin eğitimi desteklenmelidir. Hukukun temel ilkeleri açısından şeffaflık ve izlenebilirlik anayasal güvence altına alınmalı, akıllı sözleşme kodları açık kaynaklı olacak şekilde denetime açılmalıdır. Bu sürecin etkin yönetimi için akademik, teknik ve kamusal iş birliğiyle “dijital norm üretim kurulu” oluşturulmalı ve bu kurul, yönetişim standartlarını belirleyerek rehber ilkeler geliştirmelidir. Son olarak, Türkiye’nin uluslararası finans sistemine entegrasyonu açısından sınır ötesi CBDC işlemlerinde birlikte çalışabilirliği güvence altına alacak politikalar geliştirilerek küresel standartlarla uyum sağlanmalıdır. CBDC ve akıllı sözleşme entegrasyonu, finansal düzenlemeyi kökten dönüştürebilecek bir potansiyele sahiptir. Bu dönüşümün başarıyla yönetilebilmesi için \textbf{düzenleyici kurumlar} proaktif olmalı ve yeni normatif çerçeveler oluşturmalıdır. Teknolojinin hızına yetişen, ancak temel hak ve özgürlükleri gözeten bir hukukî zemin kurmak, dijital devletin meşruiyeti açısından belirleyici olacaktır.



% Kaynakça bölümü
\printbibliography


\end{document}








